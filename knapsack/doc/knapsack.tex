\documentclass[11pt]{article}
\usepackage[utf8]{inputenc}
\usepackage[czech]{babel}
\usepackage{a4wide}
\usepackage{amsmath}

\usepackage[pdfborder=0 0 0]{hyperref}

\usepackage{fancyvrb}
\usepackage{color}


\makeatletter
\def\PY@reset{\let\PY@it=\relax \let\PY@bf=\relax%
    \let\PY@ul=\relax \let\PY@tc=\relax%
    \let\PY@bc=\relax \let\PY@ff=\relax}
\def\PY@tok#1{\csname PY@tok@#1\endcsname}
\def\PY@toks#1+{\ifx\relax#1\empty\else%
    \PY@tok{#1}\expandafter\PY@toks\fi}
\def\PY@do#1{\PY@bc{\PY@tc{\PY@ul{%
    \PY@it{\PY@bf{\PY@ff{#1}}}}}}}
\def\PY#1#2{\PY@reset\PY@toks#1+\relax+\PY@do{#2}}

\def\PY@tok@gd{\def\PY@tc##1{\textcolor[rgb]{0.63,0.00,0.00}{##1}}}
\def\PY@tok@gu{\let\PY@bf=\textbf\def\PY@tc##1{\textcolor[rgb]{0.50,0.00,0.50}{##1}}}
\def\PY@tok@gt{\def\PY@tc##1{\textcolor[rgb]{0.00,0.25,0.82}{##1}}}
\def\PY@tok@gs{\let\PY@bf=\textbf}
\def\PY@tok@gr{\def\PY@tc##1{\textcolor[rgb]{1.00,0.00,0.00}{##1}}}
\def\PY@tok@cm{\let\PY@it=\textit\def\PY@tc##1{\textcolor[rgb]{0.25,0.50,0.50}{##1}}}
\def\PY@tok@vg{\def\PY@tc##1{\textcolor[rgb]{0.10,0.09,0.49}{##1}}}
\def\PY@tok@m{\def\PY@tc##1{\textcolor[rgb]{0.40,0.40,0.40}{##1}}}
\def\PY@tok@mh{\def\PY@tc##1{\textcolor[rgb]{0.40,0.40,0.40}{##1}}}
\def\PY@tok@go{\def\PY@tc##1{\textcolor[rgb]{0.50,0.50,0.50}{##1}}}
\def\PY@tok@ge{\let\PY@it=\textit}
\def\PY@tok@vc{\def\PY@tc##1{\textcolor[rgb]{0.10,0.09,0.49}{##1}}}
\def\PY@tok@il{\def\PY@tc##1{\textcolor[rgb]{0.40,0.40,0.40}{##1}}}
\def\PY@tok@cs{\let\PY@it=\textit\def\PY@tc##1{\textcolor[rgb]{0.25,0.50,0.50}{##1}}}
\def\PY@tok@cp{\def\PY@tc##1{\textcolor[rgb]{0.74,0.48,0.00}{##1}}}
\def\PY@tok@gi{\def\PY@tc##1{\textcolor[rgb]{0.00,0.63,0.00}{##1}}}
\def\PY@tok@gh{\let\PY@bf=\textbf\def\PY@tc##1{\textcolor[rgb]{0.00,0.00,0.50}{##1}}}
\def\PY@tok@ni{\let\PY@bf=\textbf\def\PY@tc##1{\textcolor[rgb]{0.60,0.60,0.60}{##1}}}
\def\PY@tok@nl{\def\PY@tc##1{\textcolor[rgb]{0.63,0.63,0.00}{##1}}}
\def\PY@tok@nn{\let\PY@bf=\textbf\def\PY@tc##1{\textcolor[rgb]{0.00,0.00,1.00}{##1}}}
\def\PY@tok@no{\def\PY@tc##1{\textcolor[rgb]{0.53,0.00,0.00}{##1}}}
\def\PY@tok@na{\def\PY@tc##1{\textcolor[rgb]{0.49,0.56,0.16}{##1}}}
\def\PY@tok@nb{\def\PY@tc##1{\textcolor[rgb]{0.00,0.50,0.00}{##1}}}
\def\PY@tok@nc{\let\PY@bf=\textbf\def\PY@tc##1{\textcolor[rgb]{0.00,0.00,1.00}{##1}}}
\def\PY@tok@nd{\def\PY@tc##1{\textcolor[rgb]{0.67,0.13,1.00}{##1}}}
\def\PY@tok@ne{\let\PY@bf=\textbf\def\PY@tc##1{\textcolor[rgb]{0.82,0.25,0.23}{##1}}}
\def\PY@tok@nf{\def\PY@tc##1{\textcolor[rgb]{0.00,0.00,1.00}{##1}}}
\def\PY@tok@si{\let\PY@bf=\textbf\def\PY@tc##1{\textcolor[rgb]{0.73,0.40,0.53}{##1}}}
\def\PY@tok@s2{\def\PY@tc##1{\textcolor[rgb]{0.73,0.13,0.13}{##1}}}
\def\PY@tok@vi{\def\PY@tc##1{\textcolor[rgb]{0.10,0.09,0.49}{##1}}}
\def\PY@tok@nt{\let\PY@bf=\textbf\def\PY@tc##1{\textcolor[rgb]{0.00,0.50,0.00}{##1}}}
\def\PY@tok@nv{\def\PY@tc##1{\textcolor[rgb]{0.10,0.09,0.49}{##1}}}
\def\PY@tok@s1{\def\PY@tc##1{\textcolor[rgb]{0.73,0.13,0.13}{##1}}}
\def\PY@tok@sh{\def\PY@tc##1{\textcolor[rgb]{0.73,0.13,0.13}{##1}}}
\def\PY@tok@sc{\def\PY@tc##1{\textcolor[rgb]{0.73,0.13,0.13}{##1}}}
\def\PY@tok@sx{\def\PY@tc##1{\textcolor[rgb]{0.00,0.50,0.00}{##1}}}
\def\PY@tok@bp{\def\PY@tc##1{\textcolor[rgb]{0.00,0.50,0.00}{##1}}}
\def\PY@tok@c1{\let\PY@it=\textit\def\PY@tc##1{\textcolor[rgb]{0.25,0.50,0.50}{##1}}}
\def\PY@tok@kc{\let\PY@bf=\textbf\def\PY@tc##1{\textcolor[rgb]{0.00,0.50,0.00}{##1}}}
\def\PY@tok@c{\let\PY@it=\textit\def\PY@tc##1{\textcolor[rgb]{0.25,0.50,0.50}{##1}}}
\def\PY@tok@mf{\def\PY@tc##1{\textcolor[rgb]{0.40,0.40,0.40}{##1}}}
\def\PY@tok@err{\def\PY@bc##1{\fcolorbox[rgb]{1.00,0.00,0.00}{1,1,1}{##1}}}
\def\PY@tok@kd{\let\PY@bf=\textbf\def\PY@tc##1{\textcolor[rgb]{0.00,0.50,0.00}{##1}}}
\def\PY@tok@ss{\def\PY@tc##1{\textcolor[rgb]{0.10,0.09,0.49}{##1}}}
\def\PY@tok@sr{\def\PY@tc##1{\textcolor[rgb]{0.73,0.40,0.53}{##1}}}
\def\PY@tok@mo{\def\PY@tc##1{\textcolor[rgb]{0.40,0.40,0.40}{##1}}}
\def\PY@tok@kn{\let\PY@bf=\textbf\def\PY@tc##1{\textcolor[rgb]{0.00,0.50,0.00}{##1}}}
\def\PY@tok@mi{\def\PY@tc##1{\textcolor[rgb]{0.40,0.40,0.40}{##1}}}
\def\PY@tok@gp{\let\PY@bf=\textbf\def\PY@tc##1{\textcolor[rgb]{0.00,0.00,0.50}{##1}}}
\def\PY@tok@o{\def\PY@tc##1{\textcolor[rgb]{0.40,0.40,0.40}{##1}}}
\def\PY@tok@kr{\let\PY@bf=\textbf\def\PY@tc##1{\textcolor[rgb]{0.00,0.50,0.00}{##1}}}
\def\PY@tok@s{\def\PY@tc##1{\textcolor[rgb]{0.73,0.13,0.13}{##1}}}
\def\PY@tok@kp{\def\PY@tc##1{\textcolor[rgb]{0.00,0.50,0.00}{##1}}}
\def\PY@tok@w{\def\PY@tc##1{\textcolor[rgb]{0.73,0.73,0.73}{##1}}}
\def\PY@tok@kt{\def\PY@tc##1{\textcolor[rgb]{0.69,0.00,0.25}{##1}}}
\def\PY@tok@ow{\let\PY@bf=\textbf\def\PY@tc##1{\textcolor[rgb]{0.67,0.13,1.00}{##1}}}
\def\PY@tok@sb{\def\PY@tc##1{\textcolor[rgb]{0.73,0.13,0.13}{##1}}}
\def\PY@tok@k{\let\PY@bf=\textbf\def\PY@tc##1{\textcolor[rgb]{0.00,0.50,0.00}{##1}}}
\def\PY@tok@se{\let\PY@bf=\textbf\def\PY@tc##1{\textcolor[rgb]{0.73,0.40,0.13}{##1}}}
\def\PY@tok@sd{\let\PY@it=\textit\def\PY@tc##1{\textcolor[rgb]{0.73,0.13,0.13}{##1}}}

\def\PYZbs{\char`\\}
\def\PYZus{\char`\_}
\def\PYZob{\char`\{}
\def\PYZcb{\char`\}}
\def\PYZca{\char`\^}
\def\PYZsh{\char`\#}
\def\PYZpc{\char`\%}
\def\PYZdl{\char`\$}
\def\PYZti{\char`\~}
% for compatibility with earlier versions
\def\PYZat{@}
\def\PYZlb{[}
\def\PYZrb{]}
\makeatother


\title{Programovací strategie: Jak pobrat odměnu}
\author{Tomáš Maršálek}
\date{6.\,října 2011}

\begin{document}
\maketitle

\section{Zadání}
Svojí nebojácností a schopnostmi jste zvítězili nad zlým čarodějem a teď si 
můžete buď odvést princeznu anebo si vybrat některý z krásných zlatých předmětů,
které kouzelník vlastní. Princezna se vám nelíbí, proto jste se rozhodl pro 
zlato. Můžete si vybrat spoustu velkých a krásných předmětů, jedinou podmínkou
je, že vybrané objekty musíte být schopen odnést v batohu, jehož nosnost 
(i vaše, koneckonců) je omezena. Pokud se vám to nepovede, nezískáte nic. 
Předměty jsou různorodé – svícny, sošky apod., každý stojí jinak a váží jinak. 
Teď se vám hodí praxe v dynamickém programování z PRO.

Vstupem je množina položek $P=\{p_1,p_2,\ldots,p_n\}$, kde položka $p_i$ má 
velikost $s_i$ a hodnotu $v_i$, batohová kapacita (tj. velikost batohu) je $C$. 
Vašim úkolem je najít podmnožinu s maximální hodnotou zjištěnou jako součet 
hodnot prvků podmnožiny takovou, aby součet velikostí prvků podmnožiny 
nepřekročil $C$ (tj. všechny vybrané položky se musí vejít do batohu). 
Velikosti položek i jejich hodnoty jsou kladná čísla do 1000.

\section{Problém}
Problém je známý jako {\sc knapsack problem}, zde přesněji 
{\sc discrete knapsack}, jelikož počet předmětů, které můžeme pobrat není 
spojitá hodnota. V opačném případě by se jednalo o {\sc liquid knapsack}, 
pro který se dá použít jednoduchý greedy algoritmus. Ještě speciálnější 
vymezení je {\sc 0-1~knapsack}, protože počet předmětů, 
které můžeme vzít od každého druhu je právě 0 nebo 1.

\section{Řešení}
Algoritmus bude na principu dynamického programování. Snažíme se maximalizovat
hodnotu batohu, tak abychom se vešli do limitu $C$.
\begin{align*}
knapsack(N, C) =\ &maxim\acute{a}ln\acute{\i}\ hodnota,\ kterou\ je\ 
mo\check{z}no\ narvat\ do\ batohu\ tak, \\
 &\check{z}e\ nep\check{r}ekro\check{c}\acute{\i}me\ mez\ C 
\end{align*}

\subsection{Myšlenka}
Optimální substruktura bude založená na jednoduché myšlence, že N-tý
předmět můžeme do batohu vložit nebo naopak ne. Ta z možností, která bude mít 
větší hodnotu nás bude zajímat. Batoh, který překročil hmotnostní limit bude 
mít hodnotu $0$.

Rekurzivní vyjádření bude vypadat zhruba následovně
\begin{align*}
krapsack(N, C) = \max(&N-tou\ polo\check{z}ku\ p\check{r}ijmout, \\
&N-tou\ polo\check{z}ku\ vyhodit)
\end{align*}

Formálně zapsáno 
\begin{align*}
knapsack(N, C) = \max(&knapsack(N - 1, C - hmotnost(N)) + hodnota(N), \\
&knapsack(N - 1, C))
\end{align*}

\begin{Verbatim}[commandchars=\\\{\}]
\PY{k}{def} \PY{n+nf}{knapsack}\PY{p}{(}\PY{n}{N}\PY{p}{,} \PY{n}{C}\PY{p}{)}\PY{p}{:}
	\PY{k}{if} \PY{n}{N} \PY{o}{==} \PY{l+m+mi}{0}\PY{p}{:}
		\PY{k}{return} \PY{l+m+mi}{0}
	\PY{n}{i} \PY{o}{=} \PY{n}{N} \PY{o}{-} \PY{l+m+mi}{1}                           \PY{c}{\PYZsh{} začínáme od nuly}
	\PY{k}{if} \PY{n}{hmotnost}\PY{p}{[}\PY{n}{i}\PY{p}{]} \PY{o}{>} \PY{n}{C}\PY{p}{:}
		\PY{k}{return} \PY{n}{knapsack}\PY{p}{(}\PY{n}{N} \PY{o}{-} \PY{l+m+mi}{1}\PY{p}{,} \PY{n}{C}\PY{p}{)}
	\PY{k}{return} \PY{n+nb}{max}\PY{p}{(}\PY{n}{knapsack}\PY{p}{(}\PY{n}{N} \PY{o}{-} \PY{l+m+mi}{1}\PY{p}{,} \PY{n}{C} \PY{o}{-} \PY{n}{hmotnost}\PY{p}{[}\PY{n}{i}\PY{p}{]}\PY{p}{)} \PY{o}{+} \PY{n}{hodnota}\PY{p}{[}\PY{n}{i}\PY{p}{]}\PY{p}{,}
	           \PY{n}{knapsack}\PY{p}{(}\PY{n}{N} \PY{o}{-} \PY{l+m+mi}{1}\PY{p}{,} \PY{n}{C}\PY{p}{)}\PY{p}{)}
\end{Verbatim}


\vspace{.5cm}
\subsection{DP provedení}
Uvedenou rekurzi vypočítáme zdola nahoru tabulární metodou (klasické dynamické
programování). 

Funkce závisí na dvou parametrech $N$ a $C$, proto složitost
bude $\Theta (NC)$. Na první pohled by se mohlo zdát, že se jedná o polynomiální
složitost, ale je třeba si uvědomit, že limit~$C$ může dosahovat exponenciálních
hodnot. {\sc Knapsack problem} je na seznamu Karpových 21 NP-úplných problémů.
\cite{wiki}
\begin{Verbatim}[commandchars=\\\{\}]
\PY{k}{def} \PY{n+nf}{knapsack}\PY{p}{(}\PY{n}{N}\PY{p}{,} \PY{n}{C}\PY{p}{)}\PY{p}{:}
	\PY{n}{tmp} \PY{o}{=} \PY{p}{[}\PY{p}{[}\PY{l+m+mi}{0} \PY{k}{for} \PY{n}{j} \PY{o+ow}{in} \PY{n+nb}{range}\PY{p}{(}\PY{n}{C} \PY{o}{+} \PY{l+m+mi}{1}\PY{p}{)}\PY{p}{]} \PY{k}{for} \PY{n}{i} \PY{o+ow}{in} \PY{n+nb}{range}\PY{p}{(}\PY{n}{N}\PY{p}{)}\PY{p}{]}  \PY{c}{\PYZsh{} DP tabulka}
	\PY{k}{for} \PY{n}{j} \PY{o+ow}{in} \PY{n+nb}{range}\PY{p}{(}\PY{n}{C} \PY{o}{+} \PY{l+m+mi}{1}\PY{p}{)}\PY{p}{:} \PY{c}{\PYZsh{} base case }
		\PY{k}{if} \PY{n}{hmotnost}\PY{p}{[}\PY{l+m+mi}{0}\PY{p}{]} \PY{o}{<}\PY{o}{=} \PY{n}{j}\PY{p}{:}
			\PY{n}{tmp}\PY{p}{[}\PY{l+m+mi}{0}\PY{p}{]}\PY{p}{[}\PY{n}{j}\PY{p}{]} \PY{o}{=} \PY{n}{hodnota}\PY{p}{[}\PY{l+m+mi}{0}\PY{p}{]}

	\PY{k}{for} \PY{n}{i} \PY{o+ow}{in} \PY{n+nb}{range}\PY{p}{(}\PY{l+m+mi}{1}\PY{p}{,} \PY{n}{N}\PY{p}{)}\PY{p}{:}
		\PY{k}{for} \PY{n}{j} \PY{o+ow}{in} \PY{n+nb}{range}\PY{p}{(}\PY{n}{C} \PY{o}{+} \PY{l+m+mi}{1}\PY{p}{)}\PY{p}{:}
			\PY{k}{if} \PY{n}{hmotnost}\PY{p}{[}\PY{n}{i}\PY{p}{]} \PY{o}{>} \PY{n}{j}\PY{p}{:}
				\PY{n}{tmp}\PY{p}{[}\PY{n}{i}\PY{p}{]}\PY{p}{[}\PY{n}{j}\PY{p}{]} \PY{o}{=} \PY{n}{tmp}\PY{p}{[}\PY{n}{i} \PY{o}{-} \PY{l+m+mi}{1}\PY{p}{]}\PY{p}{[}\PY{n}{j}\PY{p}{]}
			\PY{k}{else}\PY{p}{:}
				\PY{n}{tmp}\PY{p}{[}\PY{n}{i}\PY{p}{]}\PY{p}{[}\PY{n}{j}\PY{p}{]} \PY{o}{=} \PY{n+nb}{max}\PY{p}{(}\PY{n}{tmp}\PY{p}{[}\PY{n}{i} \PY{o}{-} \PY{l+m+mi}{1}\PY{p}{]}\PY{p}{[}\PY{n}{j} \PY{o}{-} \PY{n}{hmotnost}\PY{p}{[}\PY{n}{i}\PY{p}{]}\PY{p}{]} \PY{o}{+} \PY{n}{hodnota}\PY{p}{[}\PY{n}{i}\PY{p}{]}\PY{p}{,}
								\PY{n}{tmp}\PY{p}{[}\PY{n}{i} \PY{o}{-} \PY{l+m+mi}{1}\PY{p}{]}\PY{p}{[}\PY{n}{j}\PY{p}{]}\PY{p}{)}
	\PY{k}{return} \PY{n}{tmp}\PY{p}{[}\PY{o}{-}\PY{l+m+mi}{1}\PY{p}{]}\PY{p}{[}\PY{o}{-}\PY{l+m+mi}{1}\PY{p}{]}        \PY{c}{\PYZsh{} vrátit poslední prvek tabulky}
\end{Verbatim}


\clearpage
Položky najdeme zpětným průchodem tabulkou
\begin{Verbatim}[commandchars=\\\{\}]
	\PY{n}{polozky} \PY{o}{=} \PY{p}{[}\PY{p}{]}
	\PY{k}{while} \PY{n}{i} \PY{o}{>}\PY{o}{=} \PY{l+m+mi}{0} \PY{o+ow}{and} \PY{n}{j} \PY{o}{>}\PY{o}{=} \PY{l+m+mi}{0}\PY{p}{:}
		\PY{k}{if} \PY{n}{j} \PY{o}{>}\PY{o}{=} \PY{n}{hmotnost}\PY{p}{[}\PY{n}{i}\PY{p}{]} \PYZbs{}
		\PY{o+ow}{and} \PY{n}{tmp}\PY{p}{[}\PY{n}{i} \PY{o}{-} \PY{l+m+mi}{1}\PY{p}{]}\PY{p}{[}\PY{n}{j} \PY{o}{-} \PY{n}{hmotnost}\PY{p}{[}\PY{n}{i}\PY{p}{]}\PY{p}{]} \PY{o}{+} \PY{n}{hodnota}\PY{p}{[}\PY{n}{i}\PY{p}{]} \PY{o}{==} \PY{n}{tmp}\PY{p}{[}\PY{n}{i}\PY{p}{]}\PY{p}{[}\PY{n}{j}\PY{p}{]}\PY{p}{:}
			\PY{n}{polozky}\PY{o}{.}\PY{n}{append}\PY{p}{(}\PY{n}{i}\PY{p}{)}
			\PY{n}{j} \PY{o}{-}\PY{o}{=} \PY{n}{hmotnost}\PY{p}{[}\PY{n}{i}\PY{p}{]}
			\PY{n}{i} \PY{o}{-}\PY{o}{=} \PY{l+m+mi}{1}
		\PY{k}{elif} \PY{n}{tmp}\PY{p}{[}\PY{n}{i} \PY{o}{-} \PY{l+m+mi}{1}\PY{p}{]}\PY{p}{[}\PY{n}{j}\PY{p}{]} \PY{o}{==} \PY{n}{tmp}\PY{p}{[}\PY{n}{i}\PY{p}{]}\PY{p}{[}\PY{n}{j}\PY{p}{]}\PY{p}{:}
			\PY{n}{i} \PY{o}{-}\PY{o}{=} \PY{l+m+mi}{1}
		\PY{k}{else}\PY{p}{:}
			\PY{n}{j} \PY{o}{-}\PY{o}{=} \PY{l+m+mi}{1}
	\PY{n}{polozky} \PY{o}{=} \PY{n}{polozky}\PY{p}{[}\PY{p}{:}\PY{p}{:}\PY{o}{-}\PY{l+m+mi}{1}\PY{p}{]} \PY{c}{\PYZsh{} obrátit položky}
\end{Verbatim}


\section{Implementace}
Použitý jazyk je python 2.7.2. Implementace obsahuje generátor vstupu 
{\tt gen.py} a vlastní program {\tt knapsack.py}.

\subsection{Příklad použití}
\begin{verbatim}
# vygenerovat 100 polozek s maximalni hodnotou a maximalni vahou 1000
./gen.py 100 1000 1000 > test.data

# pouzit vygenerovana data pro knapsack s limitem 5000 (C = 5000)
cat test.data | ./knapsack.py 5000
\end{verbatim}

\vspace{1cm}
\begin{thebibliography}{0}

\bibitem[1]{aduni}
{\em Shai Simonson} \\
{\bf 02-20-01: Knapsack, Bandwidth Min. Intro: Greedy Algs.} (video) 2001 \\
\url{http://aduni.org/courses/algorithms/index.php?view=cw} \\
\url{http://video.google.com/videoplay?docid=-8586312179390822765}

\bibitem[2]{wiki}
{\em Wikipedia contributors} \\
{\bf "Knapsack problem,"\ Wikipedia, The Free Encyclopedia} \\
\url{http://en.wikipedia.org/w/index.php?title=Knapsack_problem&oldid=452859011} \\ 
(accessed October 5, 2011).

\end{thebibliography}

\end{document}
