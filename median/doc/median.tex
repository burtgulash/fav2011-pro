\documentclass[11pt]{article}
\usepackage[utf8]{inputenc}
\usepackage[czech]{babel}
\usepackage{a4wide}
\usepackage{amsmath}

\usepackage[pdfborder=0 0 0]{hyperref}
	

\usepackage{fancyvrb}
\usepackage{color}


\makeatletter
\def\PY@reset{\let\PY@it=\relax \let\PY@bf=\relax%
    \let\PY@ul=\relax \let\PY@tc=\relax%
    \let\PY@bc=\relax \let\PY@ff=\relax}
\def\PY@tok#1{\csname PY@tok@#1\endcsname}
\def\PY@toks#1+{\ifx\relax#1\empty\else%
    \PY@tok{#1}\expandafter\PY@toks\fi}
\def\PY@do#1{\PY@bc{\PY@tc{\PY@ul{%
    \PY@it{\PY@bf{\PY@ff{#1}}}}}}}
\def\PY#1#2{\PY@reset\PY@toks#1+\relax+\PY@do{#2}}

\def\PY@tok@gd{\def\PY@tc##1{\textcolor[rgb]{0.63,0.00,0.00}{##1}}}
\def\PY@tok@gu{\let\PY@bf=\textbf\def\PY@tc##1{\textcolor[rgb]{0.50,0.00,0.50}{##1}}}
\def\PY@tok@gt{\def\PY@tc##1{\textcolor[rgb]{0.00,0.25,0.82}{##1}}}
\def\PY@tok@gs{\let\PY@bf=\textbf}
\def\PY@tok@gr{\def\PY@tc##1{\textcolor[rgb]{1.00,0.00,0.00}{##1}}}
\def\PY@tok@cm{\let\PY@it=\textit\def\PY@tc##1{\textcolor[rgb]{0.25,0.50,0.50}{##1}}}
\def\PY@tok@vg{\def\PY@tc##1{\textcolor[rgb]{0.10,0.09,0.49}{##1}}}
\def\PY@tok@m{\def\PY@tc##1{\textcolor[rgb]{0.40,0.40,0.40}{##1}}}
\def\PY@tok@mh{\def\PY@tc##1{\textcolor[rgb]{0.40,0.40,0.40}{##1}}}
\def\PY@tok@go{\def\PY@tc##1{\textcolor[rgb]{0.50,0.50,0.50}{##1}}}
\def\PY@tok@ge{\let\PY@it=\textit}
\def\PY@tok@vc{\def\PY@tc##1{\textcolor[rgb]{0.10,0.09,0.49}{##1}}}
\def\PY@tok@il{\def\PY@tc##1{\textcolor[rgb]{0.40,0.40,0.40}{##1}}}
\def\PY@tok@cs{\let\PY@it=\textit\def\PY@tc##1{\textcolor[rgb]{0.25,0.50,0.50}{##1}}}
\def\PY@tok@cp{\def\PY@tc##1{\textcolor[rgb]{0.74,0.48,0.00}{##1}}}
\def\PY@tok@gi{\def\PY@tc##1{\textcolor[rgb]{0.00,0.63,0.00}{##1}}}
\def\PY@tok@gh{\let\PY@bf=\textbf\def\PY@tc##1{\textcolor[rgb]{0.00,0.00,0.50}{##1}}}
\def\PY@tok@ni{\let\PY@bf=\textbf\def\PY@tc##1{\textcolor[rgb]{0.60,0.60,0.60}{##1}}}
\def\PY@tok@nl{\def\PY@tc##1{\textcolor[rgb]{0.63,0.63,0.00}{##1}}}
\def\PY@tok@nn{\let\PY@bf=\textbf\def\PY@tc##1{\textcolor[rgb]{0.00,0.00,1.00}{##1}}}
\def\PY@tok@no{\def\PY@tc##1{\textcolor[rgb]{0.53,0.00,0.00}{##1}}}
\def\PY@tok@na{\def\PY@tc##1{\textcolor[rgb]{0.49,0.56,0.16}{##1}}}
\def\PY@tok@nb{\def\PY@tc##1{\textcolor[rgb]{0.00,0.50,0.00}{##1}}}
\def\PY@tok@nc{\let\PY@bf=\textbf\def\PY@tc##1{\textcolor[rgb]{0.00,0.00,1.00}{##1}}}
\def\PY@tok@nd{\def\PY@tc##1{\textcolor[rgb]{0.67,0.13,1.00}{##1}}}
\def\PY@tok@ne{\let\PY@bf=\textbf\def\PY@tc##1{\textcolor[rgb]{0.82,0.25,0.23}{##1}}}
\def\PY@tok@nf{\def\PY@tc##1{\textcolor[rgb]{0.00,0.00,1.00}{##1}}}
\def\PY@tok@si{\let\PY@bf=\textbf\def\PY@tc##1{\textcolor[rgb]{0.73,0.40,0.53}{##1}}}
\def\PY@tok@s2{\def\PY@tc##1{\textcolor[rgb]{0.73,0.13,0.13}{##1}}}
\def\PY@tok@vi{\def\PY@tc##1{\textcolor[rgb]{0.10,0.09,0.49}{##1}}}
\def\PY@tok@nt{\let\PY@bf=\textbf\def\PY@tc##1{\textcolor[rgb]{0.00,0.50,0.00}{##1}}}
\def\PY@tok@nv{\def\PY@tc##1{\textcolor[rgb]{0.10,0.09,0.49}{##1}}}
\def\PY@tok@s1{\def\PY@tc##1{\textcolor[rgb]{0.73,0.13,0.13}{##1}}}
\def\PY@tok@sh{\def\PY@tc##1{\textcolor[rgb]{0.73,0.13,0.13}{##1}}}
\def\PY@tok@sc{\def\PY@tc##1{\textcolor[rgb]{0.73,0.13,0.13}{##1}}}
\def\PY@tok@sx{\def\PY@tc##1{\textcolor[rgb]{0.00,0.50,0.00}{##1}}}
\def\PY@tok@bp{\def\PY@tc##1{\textcolor[rgb]{0.00,0.50,0.00}{##1}}}
\def\PY@tok@c1{\let\PY@it=\textit\def\PY@tc##1{\textcolor[rgb]{0.25,0.50,0.50}{##1}}}
\def\PY@tok@kc{\let\PY@bf=\textbf\def\PY@tc##1{\textcolor[rgb]{0.00,0.50,0.00}{##1}}}
\def\PY@tok@c{\let\PY@it=\textit\def\PY@tc##1{\textcolor[rgb]{0.25,0.50,0.50}{##1}}}
\def\PY@tok@mf{\def\PY@tc##1{\textcolor[rgb]{0.40,0.40,0.40}{##1}}}
\def\PY@tok@err{\def\PY@bc##1{\fcolorbox[rgb]{1.00,0.00,0.00}{1,1,1}{##1}}}
\def\PY@tok@kd{\let\PY@bf=\textbf\def\PY@tc##1{\textcolor[rgb]{0.00,0.50,0.00}{##1}}}
\def\PY@tok@ss{\def\PY@tc##1{\textcolor[rgb]{0.10,0.09,0.49}{##1}}}
\def\PY@tok@sr{\def\PY@tc##1{\textcolor[rgb]{0.73,0.40,0.53}{##1}}}
\def\PY@tok@mo{\def\PY@tc##1{\textcolor[rgb]{0.40,0.40,0.40}{##1}}}
\def\PY@tok@kn{\let\PY@bf=\textbf\def\PY@tc##1{\textcolor[rgb]{0.00,0.50,0.00}{##1}}}
\def\PY@tok@mi{\def\PY@tc##1{\textcolor[rgb]{0.40,0.40,0.40}{##1}}}
\def\PY@tok@gp{\let\PY@bf=\textbf\def\PY@tc##1{\textcolor[rgb]{0.00,0.00,0.50}{##1}}}
\def\PY@tok@o{\def\PY@tc##1{\textcolor[rgb]{0.40,0.40,0.40}{##1}}}
\def\PY@tok@kr{\let\PY@bf=\textbf\def\PY@tc##1{\textcolor[rgb]{0.00,0.50,0.00}{##1}}}
\def\PY@tok@s{\def\PY@tc##1{\textcolor[rgb]{0.73,0.13,0.13}{##1}}}
\def\PY@tok@kp{\def\PY@tc##1{\textcolor[rgb]{0.00,0.50,0.00}{##1}}}
\def\PY@tok@w{\def\PY@tc##1{\textcolor[rgb]{0.73,0.73,0.73}{##1}}}
\def\PY@tok@kt{\def\PY@tc##1{\textcolor[rgb]{0.69,0.00,0.25}{##1}}}
\def\PY@tok@ow{\let\PY@bf=\textbf\def\PY@tc##1{\textcolor[rgb]{0.67,0.13,1.00}{##1}}}
\def\PY@tok@sb{\def\PY@tc##1{\textcolor[rgb]{0.73,0.13,0.13}{##1}}}
\def\PY@tok@k{\let\PY@bf=\textbf\def\PY@tc##1{\textcolor[rgb]{0.00,0.50,0.00}{##1}}}
\def\PY@tok@se{\let\PY@bf=\textbf\def\PY@tc##1{\textcolor[rgb]{0.73,0.40,0.13}{##1}}}
\def\PY@tok@sd{\let\PY@it=\textit\def\PY@tc##1{\textcolor[rgb]{0.73,0.13,0.13}{##1}}}

\def\PYZbs{\char`\\}
\def\PYZus{\char`\_}
\def\PYZob{\char`\{}
\def\PYZcb{\char`\}}
\def\PYZca{\char`\^}
\def\PYZsh{\char`\#}
\def\PYZpc{\char`\%}
\def\PYZdl{\char`\$}
\def\PYZti{\char`\~}
% for compatibility with earlier versions
\def\PYZat{@}
\def\PYZlb{[}
\def\PYZrb{]}
\makeatother




\title{Programovací strategie: Medián}
\author{Tomáš Maršálek}
\date{6.\,října 2011}

%% -----------------------
%% preamble end



\begin{document}
\maketitle

\section{Zadání}
Najděte v dostupné literatuře nebo vymyslete co nejlepší algoritmus pro výpočet mediánu. Nezapomeňte na citaci zdrojů. Kritéria kvality v sestupném pořadí jsou: výpočetní složitost, jednoduchost a implementační nenáročnost, paměťová spotřeba.

\section{Problém}
Máme posloupnost $N$ porovnatelných prvků, ve které hledáme K-tý nejmenší prvek.
Speciálním případem je medián, pro který $K = N/2$. Triviálním řešením je 
seřazení posloupnosti a odpočítání K-tého prvku. Pro obecný případ tedy 
máme horní hranici $O(N \log N)$. Pokud pro danou datovou množinu existuje lepší
řadící algoritmus (např. radix sort), jsme schopni najít medián v lineárním
čase.

\section{Obecné řešení}
Algoritmus quickselect (Hoare, 1962) je velmi podobný quicksortu. S ním sdílí
stejnou rozdělující proceduru a složitost nejhoršího případu $O(N^2)$, 
očekávaná složitost u běžného případu je ale $O(N)$ oproti $O(N \log N)$ 
u quicksortu.

\section{Implementace}
\subsection{Rozdělení}
Rozdělovací funkce určí prvek zvaný pivot a uspořádá prvky posloupnosti tak,
že všechny prvky menší než pivot budou vlevo a všechny větší vpravo od pivotu. 
Pole pak vypadá následovně:
\vspace{1cm}

%% Partition figure %%
\setlength{\unitlength}{.8mm}
\begin{center}
\begin{picture}(0, 0)

\put(0,0){\line(1, 0){50}}
\put(0,0){\line(-1, 0){50}}

\put(0,10){\line(1, 0){50}}
\put(0,10){\line(-1, 0){50}}

\put(-50, 0){\line(0, 1){10}}
\put(50, 0){\line(0, 1){10}}

\put(-10, 0){\line(0, 1){10}}
\put(10, 0){\line(0, 1){10}}

\put(-5, 4){$pivot$}
\put(-45, 4){$\{ar[x] < pivot \}$}
\put(15, 4){$\{ar[x] \geq pivot \}$}
\end{picture}
\end{center}

\clearpage
\begin{Verbatim}[commandchars=\\\{\}]
\PY{k}{def} \PY{n+nf}{rozdelit}\PY{p}{(}\PY{n}{ar}\PY{p}{,} \PY{n}{levy}\PY{p}{,} \PY{n}{pravy}\PY{p}{,} \PY{n}{p}\PY{p}{)}\PY{p}{:}
	\PY{n}{pivot} \PY{o}{=} \PY{n}{ar}\PY{p}{[}\PY{n}{p}\PY{p}{]}
	\PY{n}{i}\PY{p}{,} \PY{n}{j} \PY{o}{=} \PY{n}{levy}\PY{p}{,} \PY{n}{levy}
	\PY{n}{ar}\PY{p}{[}\PY{n}{pravy}\PY{p}{]}\PY{p}{,} \PY{n}{ar}\PY{p}{[}\PY{n}{p}\PY{p}{]} \PY{o}{=} \PY{n}{ar}\PY{p}{[}\PY{n}{p}\PY{p}{]}\PY{p}{,} \PY{n}{ar}\PY{p}{[}\PY{n}{pravy}\PY{p}{]}
	\PY{k}{while} \PY{n}{i} \PY{o}{<}\PY{o}{=} \PY{n}{pravy}\PY{p}{:}
		\PY{k}{if} \PY{n}{ar}\PY{p}{[}\PY{n}{i}\PY{p}{]} \PY{o}{<} \PY{n}{pivot}\PY{p}{:}
			\PY{n}{ar}\PY{p}{[}\PY{n}{i}\PY{p}{]}\PY{p}{,} \PY{n}{ar}\PY{p}{[}\PY{n}{j}\PY{p}{]} \PY{o}{=} \PY{n}{ar}\PY{p}{[}\PY{n}{j}\PY{p}{]}\PY{p}{,} \PY{n}{ar}\PY{p}{[}\PY{n}{i}\PY{p}{]}
			\PY{n}{j} \PY{o}{+}\PY{o}{=} \PY{l+m+mi}{1}
		\PY{n}{i} \PY{o}{+}\PY{o}{=} \PY{l+m+mi}{1}
	\PY{n}{ar}\PY{p}{[}\PY{n}{pravy}\PY{p}{]}\PY{p}{,} \PY{n}{ar}\PY{p}{[}\PY{n}{j}\PY{p}{]} \PY{o}{=} \PY{n}{ar}\PY{p}{[}\PY{n}{j}\PY{p}{]}\PY{p}{,} \PY{n}{ar}\PY{p}{[}\PY{n}{pravy}\PY{p}{]}
	\PY{k}{return} \PY{n}{j}
\end{Verbatim}


\subsection{Quickselect}
Algoritmus nejprve vybere náhodný prvek jako pivot a podle něj rozdělí pole.
O pivotu teď můžeme s jistotou tvrdit, že kdyby bylo pole seřázené, byl by na 
stejné pozici.  Pokud se zadařilo a pozice pivotu se shoduje s pozicí 
hledaného prvku K, máme hotovo, pivot je hledaným prvkem. Pokud ne, pak 
rekurzivně opakujeme quickselect na tu část pole, ve které prvek definitivně 
bude. V případě pole o jednom prvku $(levy = pravy)$ vrátíme tento prvek.
\begin{Verbatim}[commandchars=\\\{\}]
\PY{k}{def} \PY{n+nf}{quickselect}\PY{p}{(}\PY{n}{ar}\PY{p}{,} \PY{n}{levy}\PY{p}{,} \PY{n}{pravy}\PY{p}{,} \PY{n}{K}\PY{p}{)}\PY{p}{:}
	\PY{k}{if} \PY{n}{levy} \PY{o}{==} \PY{n}{pravy}\PY{p}{:}
		\PY{k}{return} \PY{n}{ar}\PY{p}{[}\PY{n}{levy}\PY{p}{]}
	\PY{n}{nahodnyIndex} \PY{o}{=} \PY{n}{random}\PY{o}{.}\PY{n}{randint}\PY{p}{(}\PY{n}{levy}\PY{p}{,} \PY{n}{pravy}\PY{p}{)}
	\PY{n}{pivotIndex} \PY{o}{=} \PY{n}{rozdelit}\PY{p}{(}\PY{n}{ar}\PY{p}{,} \PY{n}{levy}\PY{p}{,} \PY{n}{pravy}\PY{p}{,} \PY{n}{nahodnyIndex}\PY{p}{)}
	\PY{k}{if} \PY{n}{pivotIndex} \PY{o}{==} \PY{n}{K}\PY{p}{:}
		\PY{k}{return} \PY{n}{ar}\PY{p}{[}\PY{n}{pivotIndex}\PY{p}{]}
	\PY{k}{if} \PY{n}{pivotIndex} \PY{o}{<} \PY{n}{K}\PY{p}{:}
		\PY{k}{return} \PY{n}{quickselect}\PY{p}{(}\PY{n}{ar}\PY{p}{,} \PY{n}{pivotIndex} \PY{o}{+} \PY{l+m+mi}{1}\PY{p}{,} \PY{n}{pravy}\PY{p}{,} \PY{n}{K}\PY{p}{)}
	\PY{k}{else}\PY{p}{:}
		\PY{k}{return} \PY{n}{quickselect}\PY{p}{(}\PY{n}{ar}\PY{p}{,} \PY{n}{levy}\PY{p}{,} \PY{n}{pivotIndex} \PY{o}{-} \PY{l+m+mi}{1}\PY{p}{,} \PY{n}{K}\PY{p}{)}
\end{Verbatim}


\subsection{Optimalizace}
Po odstranění koncové rekurze se kód podobá binárnímu vyhledávání.
\begin{Verbatim}[commandchars=\\\{\}]
\PY{k}{def} \PY{n+nf}{quickselect}\PY{p}{(}\PY{n}{ar}\PY{p}{,} \PY{n}{levy}\PY{p}{,} \PY{n}{pravy}\PY{p}{,} \PY{n}{K}\PY{p}{)}\PY{p}{:}
	\PY{k}{while} \PY{n}{levy} \PY{o}{!=} \PY{n}{pravy}\PY{p}{:}
		\PY{n}{nahodnyIndex} \PY{o}{=} \PY{n}{random}\PY{o}{.}\PY{n}{randint}\PY{p}{(}\PY{n}{levy}\PY{p}{,} \PY{n}{pravy}\PY{p}{)}
		\PY{n}{pivotIndex} \PY{o}{=} \PY{n}{rozdelit}\PY{p}{(}\PY{n}{ar}\PY{p}{,} \PY{n}{levy}\PY{p}{,} \PY{n}{pravy}\PY{p}{,} \PY{n}{nahodnyIndex}\PY{p}{)}
		\PY{k}{if} \PY{n}{pivotIndex} \PY{o}{==} \PY{n}{K}\PY{p}{:}
			\PY{k}{return} \PY{n}{ar}\PY{p}{[}\PY{n}{pivotIndex}\PY{p}{]}
		\PY{k}{if} \PY{n}{pivotIndex} \PY{o}{<} \PY{n}{K}\PY{p}{:}
			\PY{n}{levy} \PY{o}{=} \PY{n}{pivotIndex} \PY{o}{+} \PY{l+m+mi}{1}
		\PY{k}{else}\PY{p}{:}
			\PY{n}{pravy} \PY{o}{=} \PY{n}{pivotIndex} \PY{o}{-} \PY{l+m+mi}{1}
	\PY{k}{return} \PY{n}{ar}\PY{p}{[}\PY{n}{levy}\PY{p}{]}
\end{Verbatim}



\section{Časová náročnost}
Složitost algoritmu je závislá na výběru pivotu. Nejjednodušším řešením je
výběr fixního prvku, například prvku nejvíce vlevo. Pak ale při seřazeném poli 
složitost degeneruje na případ $O(N^2)$. Randomizovaný výběr alespoň ochrání 
před případem seřazeného pole, ale stále může s malou pravděpodobností nastat 
degenerující případ.

\subsection{Ideální případ}
Budeme-li předpokládat, že vybraný pivot je v každé iteraci mediánem posloupnosti, pak časová složitost bude
\begin{align*}
T(N) &= T(N/2) + N \\
&= T(N/4) + N + N/2 \\
&= T(N/8) + N + N/2 + N/4 \\
&= T(N/2^i) + (N + N/2 + N/4 + \ldots + N/2^i) \\
&< O(1) + 2N \\
&= O(N)
\end{align*}

Obdobně bude-li výběr pivotu v každé iteraci rozdělovat posloupnost na dvě části
podle nějakého zlomku, který nezávisí na N (např. $1/2~|~1/2$ nebo 
$75/100~|~25/100$ nebo $pN~|~(1~-~p)N$ pro $0~<~p~<~1$, 
ale ne $(N~-~1)/N~|~1/N$), bude složitost pro dostatečně velkou konstantu 
stále lineární
\begin{align*}
T(N) &= T(pN) + N \\
&= T(p^2 N) + N + p N \\
&= T(p^i N) + (N + p N + \ldots + p^i N) \\
&= O(1) + N \frac{1 - p^{1 - \log _p N}}{1 - p} \\
&< O(1) + N \frac{1}{1 - p} \\
&= O(N)
\end{align*}

\subsection{Nejhorší případ}
Nastane, pokud každé rozdělení rozdělí posloupnost na jeden prvek a $N~-~1$
prvků
\begin{align*}
T(N) &= T(N - 1) + N \\
&= T(N - 2) + (N) + (N - 1) \\
&= T(N - 3) + (N) + (N - 2) + (N - 3) \\
&= O(1) + \displaystyle\sum\limits_{i = 0}^{N} (N - i) \\
&= O(N^2)
\end{align*}
Ten ale v praxi potkáme s velmi malou pravděpodobností.

\section{Vylepšení}
Výběr pivotu je klíčovým prvkem, který určuje složitost celého algoritmu. Existuje
však metoda pro výběr pivotu takového, že zaručí složitost 
nejhoršího případu $O(N)$, známá jako {\sc Median of Medians} nebo 
{\sc BFPRT (Blum, Floyd, Pratt, Rivest, Tarjan, 1973)} \cite{ocwmit}.

\section{Dodatek k implementaci}
Program {\tt median.py} je v jazyce python 2.7.2. Povolená vstupní data jsou 
pouze celá čísla.  Program slouží pouze jako demonstrace, proto neplánuji 
rozšíření pro jiné datové typy.

\subsection{Příklad použití pro bash}
\begin{verbatim}
# vygenerovani zkusebnich dat
> test.data; for i in $(seq 1 10000); do echo $RANDOM >> test.data; done

# vypocet medianu vzorovych dat
cat test.data | ./median.py

# overeni vysledku (serazeni a vypsani n/2 teho radku)
cat test.data | sort -n | head -n $(($(wc -l < test.data)/2)) | tail -1

\end{verbatim}

\begin{thebibliography}{3}
\bibitem[1]{CLRS}
{\em Cormen, Thomas H.; Leiserson, Charles E.; 
	Rivest, Ronald L.; Stein, Clifford} \\
{\bf Introduction to Algorithms (2nd ed.)} \\
	MIT Press and McGraw-Hill, 2001

\bibitem[2]{ocwmit}
{\em Prof. Erik Demaine, Prof. Charles Leiserson} \\
{\bf Introduction to Algorithms (SMA 5503), 
	Lecture 6: Order Statistics, Median} (video) 2005 \\
\url{http://ocw.mit.edu/courses/electrical-engineering-and-computer-science/6-046j-introduction-to-algorithms-sma-5503-fall-2005/video-lectures/lecture-6-order-statistics-median/}

\bibitem[3]{cs61b}
{\em Jonathan Shewchuk} \\
{\bf CS 61B Lecture 32: Sorting III} (video) 2006 \\
\url{http://www.youtube.com/watch?v=Y6LOLpxg6Dc}


\end{thebibliography}

\end{document}
