\documentclass[11pt]{article}
\usepackage[utf8]{inputenc}
\usepackage[czech]{babel}
\usepackage{a4wide}
\usepackage{amsmath}


\include{pygments}


\title{Programovací strategie:  Maximalizujte svůj zisk}
\author{Tomáš Maršálek}
\date{6.\,října 2011}

%% -----------------------------------------


\begin{document}
\maketitle

\section{Zadání}
Se soutěžemi se v televizi roztrhl pytel. Je na čase, abyste si vydělali také 
vy. Otázky jsou pro vás naprosto triviální a přichází čas vybrat si odměnu. 
Přivedou před vás n asistentek. Asistentky se drží za ruce a každá má na krku 
pověšenou ceduli s celočíselnou sumou, kterou získáte (kladné číslo) nebo 
naopak ztratíte (záporné číslo), když si tuto asistentku vyberete. Můžete si 
vybrat jakýkoliv počet asistentek, ale jejich spojené ruce smíte přerušit max. 
na dvou místech, tedy posloupnost odměn, kterou si vyberete, musí být spojitá, 
ale může být libovolně dlouhá. Asistentky nezískáváte, pouze sumy výher, 
takže jiná než finanční kritéria není třeba brát v úvahu.

\section{Naivní řešení}
Jednoduše vyzkoušíme všechny souvislé podposloupnosti a vybereme tu s nejvyšší
sumou. Tedy pro všechny možné délky podposloupnosti a pro všechny jejich možné 
posuny zkoušíme, jestli je jejich suma lepší než dosavadní.
\begin{Verbatim}[commandchars=\\\{\}]
\PY{k}{def} \PY{n+nf}{naivniZisk}\PY{p}{(}\PY{n}{h}\PY{p}{)}\PY{p}{:}
	\PY{n}{n} \PY{o}{=} \PY{n+nb}{len}\PY{p}{(}\PY{n}{h}\PY{p}{)}
	\PY{n}{max\PYZus{}zisk}\PY{p}{,} \PY{n}{max\PYZus{}i}\PY{p}{,} \PY{n}{max\PYZus{}delka} \PY{o}{=} \PY{l+m+mi}{0}\PY{p}{,} \PY{l+m+mi}{0}\PY{p}{,} \PY{l+m+mi}{0}
	\PY{n}{pocatecni\PYZus{}zisk} \PY{o}{=} \PY{l+m+mi}{0}
	\PY{k}{for} \PY{n}{delka} \PY{o+ow}{in} \PY{n+nb}{range}\PY{p}{(}\PY{l+m+mi}{1}\PY{p}{,} \PY{n}{n} \PY{o}{+} \PY{l+m+mi}{1}\PY{p}{)}\PY{p}{:}
		\PY{n}{pocatecni\PYZus{}zisk} \PY{o}{+}\PY{o}{=} \PY{n}{h}\PY{p}{[}\PY{n}{delka} \PY{o}{-} \PY{l+m+mi}{1}\PY{p}{]}
		\PY{n}{zisk} \PY{o}{=} \PY{n}{pocatecni\PYZus{}zisk}

		\PY{k}{for} \PY{n}{i} \PY{o+ow}{in} \PY{n+nb}{range}\PY{p}{(}\PY{n}{n} \PY{o}{-} \PY{n}{delka}\PY{p}{)}\PY{p}{:}
			\PY{k}{if} \PY{n}{zisk} \PY{o}{>} \PY{n}{max\PYZus{}zisk}\PY{p}{:}
				\PY{n}{max\PYZus{}zisk}\PY{p}{,} \PY{n}{max\PYZus{}i}\PY{p}{,} \PY{n}{max\PYZus{}delka} \PY{o}{=} \PY{n}{zisk}\PY{p}{,} \PY{n}{i}\PY{p}{,} \PY{n}{delka}
			\PY{n}{zisk} \PY{o}{+}\PY{o}{=} \PY{n}{h}\PY{p}{[}\PY{n}{delka} \PY{o}{+} \PY{n}{i}\PY{p}{]} \PY{o}{-} \PY{n}{h}\PY{p}{[}\PY{n}{i}\PY{p}{]}

		\PY{k}{if} \PY{n}{zisk} \PY{o}{>} \PY{n}{max\PYZus{}zisk}\PY{p}{:}
			\PY{n}{max\PYZus{}zisk}\PY{p}{,} \PY{n}{max\PYZus{}i}\PY{p}{,} \PY{n}{max\PYZus{}delka} \PY{o}{=} \PY{n}{zisk}\PY{p}{,} \PY{n}{n} \PY{o}{-} \PY{n}{delka}\PY{p}{,} \PY{n}{delka}
	\PY{k}{return} \PY{n}{max\PYZus{}zisk}\PY{p}{,} \PY{n}{max\PYZus{}i}\PY{p}{,} \PY{n}{max\PYZus{}delka}
\end{Verbatim}


\subsection{Složitost}
Máme dvojitě vnořený cyklus, očekáváme kvadratickou složitost.
\begin{align*}
T(N) &= (\Theta(1) + (N - 1)) + (\Theta(1) + (N - 2)) + \ldots + (\Theta(1) + (N - N)) \\
&= \Theta(N) + N^2 - \frac{N}{2}(N + 1) \\
&= \Theta (N^2)
\end{align*}

\vspace{1cm}

\section{Dynamické programování}
Definujme funkci zisk, kterou chceme maximalizovat pro posloupnost délky N
$$zisk(N) = nejvy\check{s}\check{s}\acute{\i}\ suma\ v\ cel\acute{e}\  
posloupnosti \\ $$

\setlength{\unitlength}{.6mm}
\begin{picture}(150, 15)(-30, 0)

\put(-30, 3.5){\em N:}
\thinlines
\put(0, 0){\line(1, 0){150}}
\put(0, 10){\line(1, 0){150}}
\put(0, 0){\line(0, 1){10}}
\put(150, 0){\line(0, 1){10}}

\linethickness{.5mm}
% zisk(N)
\put(35, 3.5){\em zisk(N)}
\put(20, 0){\line(1, 0){50}}
\put(20, 10){\line(1, 0){50}}
\put(20, 0){\line(0, 1){10}}
\put(70, 0){\line(0, 1){10}}

\end{picture}
 \\

\vspace{1cm}

Abychom našli algoritmus založený na dynamickém programování, hledáme rekurzivní
vyjádření této funkce. Dodefinujeme ještě pomocné funkce

\begin{align*}
krajni\_zisk(N) &= nejvy\check{s}\check{s}\acute{\i}\ zisk\ z\ 
posloupnosti\ u\ prav\acute{e}ho\ kraje \\
h(N) &= hodnota\ N-t\acute{e}ho\ prvku
\end{align*}
\mbox{}

\begin{picture}(150, 15)(-30, 0)

%% figure 1 %%
\put(-30, 3.5){\em N:}

\thinlines
\put(0, 0){\line(1, 0){150}}
\put(0, 10){\line(1, 0){150}}
\put(0, 0){\line(0, 1){10}}
\put(150, 0){\line(0, 1){10}}

\linethickness{.5mm}
% zisk(N)
\put(35, 3.5){\em zisk(N)}
\put(20, 0){\line(1, 0){50}}
\put(20, 10){\line(1, 0){50}}
\put(20, 0){\line(0, 1){10}}
\put(70, 0){\line(0, 1){10}}

% krajni_max(N)
\put(96, 3.5){\em krajni\_zisk(N)}
\put(80, 0){\line(1, 0){70}}
\put(80, 10){\line(1, 0){70}}
\put(80, 0){\line(0, 1){10}}
\put(150, 0){\line(0, 1){10}}

\end{picture}
\mbox{}

\begin{picture}(150, 15)(-30, 0)

%% figure 2 %%
\put(-30, 3.5){\em N + 1:}
\thinlines
\put(0, 0){\line(1, 0){150}}
\put(0, 10){\line(1, 0){150}}
\put(0, 0){\line(0, 1){10}}
\put(150, 0){\line(0, 1){10}}

\linethickness{.5mm}
% zisk(N)
\put(35, 3.5){\em zisk(N)}
\put(20, 0){\line(1, 0){50}}
\put(20, 10){\line(1, 0){50}}
\put(20, 0){\line(0, 1){10}}
\put(70, 0){\line(0, 1){10}}

% krajni_max(N)
\put(96, 3.5){\em krajni\_zisk(N)}
\put(80, 0){\line(1, 0){70}}
\put(80, 10){\line(1, 0){70}}
\put(80, 0){\line(0, 1){10}}
\put(150, 0){\line(0, 1){10}}

% N + 1 hodnota
\put(155, 3.5){\em h(N + 1)}
\put(150, 0){\line(1, 0){35}}
\put(150, 10){\line(1, 0){35}}
\put(185, 0){\line(0, 1){10}}

\end{picture}


\mbox{}
\vspace{1cm}

\clearpage
Vyjádříme-li $zisk(N + 1)$ pomocí $zisk(N)$, mohou nastat tři případy.
\begin{enumerate}
\item krajní zisk nám vůbec nepomohl a $zisk(N + 1) = zisk(N)$ \\
\documentclass[11pt]{article}
\usepackage[utf8]{inputenc}
\usepackage[czech]{babel}
\usepackage{a4wide}
\usepackage{amsmath}


\include{pygments}


\title{Programovací strategie:  Maximalizujte svůj zisk}
\author{Tomáš Maršálek}
\date{6.\,října 2011}

%% -----------------------------------------


\begin{document}
\maketitle

\section{Zadání}
Se soutěžemi se v televizi roztrhl pytel. Je na čase, abyste si vydělali také 
vy. Otázky jsou pro vás naprosto triviální a přichází čas vybrat si odměnu. 
Přivedou před vás n asistentek. Asistentky se drží za ruce a každá má na krku 
pověšenou ceduli s celočíselnou sumou, kterou získáte (kladné číslo) nebo 
naopak ztratíte (záporné číslo), když si tuto asistentku vyberete. Můžete si 
vybrat jakýkoliv počet asistentek, ale jejich spojené ruce smíte přerušit max. 
na dvou místech, tedy posloupnost odměn, kterou si vyberete, musí být spojitá, 
ale může být libovolně dlouhá. Asistentky nezískáváte, pouze sumy výher, 
takže jiná než finanční kritéria není třeba brát v úvahu.

\section{Naivní řešení}
Jednoduše vyzkoušíme všechny souvislé podposloupnosti a vybereme tu s nejvyšší
sumou. Tedy pro všechny možné délky podposloupnosti a pro všechny jejich možné 
posuny zkoušíme, jestli je jejich suma lepší než dosavadní.
\begin{Verbatim}[commandchars=\\\{\}]
\PY{k}{def} \PY{n+nf}{naivniZisk}\PY{p}{(}\PY{n}{h}\PY{p}{)}\PY{p}{:}
	\PY{n}{n} \PY{o}{=} \PY{n+nb}{len}\PY{p}{(}\PY{n}{h}\PY{p}{)}
	\PY{n}{max\PYZus{}zisk}\PY{p}{,} \PY{n}{max\PYZus{}i}\PY{p}{,} \PY{n}{max\PYZus{}delka} \PY{o}{=} \PY{l+m+mi}{0}\PY{p}{,} \PY{l+m+mi}{0}\PY{p}{,} \PY{l+m+mi}{0}
	\PY{n}{pocatecni\PYZus{}zisk} \PY{o}{=} \PY{l+m+mi}{0}
	\PY{k}{for} \PY{n}{delka} \PY{o+ow}{in} \PY{n+nb}{range}\PY{p}{(}\PY{l+m+mi}{1}\PY{p}{,} \PY{n}{n} \PY{o}{+} \PY{l+m+mi}{1}\PY{p}{)}\PY{p}{:}
		\PY{n}{pocatecni\PYZus{}zisk} \PY{o}{+}\PY{o}{=} \PY{n}{h}\PY{p}{[}\PY{n}{delka} \PY{o}{-} \PY{l+m+mi}{1}\PY{p}{]}
		\PY{n}{zisk} \PY{o}{=} \PY{n}{pocatecni\PYZus{}zisk}

		\PY{k}{for} \PY{n}{i} \PY{o+ow}{in} \PY{n+nb}{range}\PY{p}{(}\PY{n}{n} \PY{o}{-} \PY{n}{delka}\PY{p}{)}\PY{p}{:}
			\PY{k}{if} \PY{n}{zisk} \PY{o}{>} \PY{n}{max\PYZus{}zisk}\PY{p}{:}
				\PY{n}{max\PYZus{}zisk}\PY{p}{,} \PY{n}{max\PYZus{}i}\PY{p}{,} \PY{n}{max\PYZus{}delka} \PY{o}{=} \PY{n}{zisk}\PY{p}{,} \PY{n}{i}\PY{p}{,} \PY{n}{delka}
			\PY{n}{zisk} \PY{o}{+}\PY{o}{=} \PY{n}{h}\PY{p}{[}\PY{n}{delka} \PY{o}{+} \PY{n}{i}\PY{p}{]} \PY{o}{-} \PY{n}{h}\PY{p}{[}\PY{n}{i}\PY{p}{]}

		\PY{k}{if} \PY{n}{zisk} \PY{o}{>} \PY{n}{max\PYZus{}zisk}\PY{p}{:}
			\PY{n}{max\PYZus{}zisk}\PY{p}{,} \PY{n}{max\PYZus{}i}\PY{p}{,} \PY{n}{max\PYZus{}delka} \PY{o}{=} \PY{n}{zisk}\PY{p}{,} \PY{n}{n} \PY{o}{-} \PY{n}{delka}\PY{p}{,} \PY{n}{delka}
	\PY{k}{return} \PY{n}{max\PYZus{}zisk}\PY{p}{,} \PY{n}{max\PYZus{}i}\PY{p}{,} \PY{n}{max\PYZus{}delka}
\end{Verbatim}


\subsection{Složitost}
Máme dvojitě vnořený cyklus, očekáváme kvadratickou složitost.
\begin{align*}
T(N) &= (\Theta(1) + (N - 1)) + (\Theta(1) + (N - 2)) + \ldots + (\Theta(1) + (N - N)) \\
&= \Theta(N) + N^2 - \frac{N}{2}(N + 1) \\
&= \Theta (N^2)
\end{align*}

\vspace{1cm}

\section{Dynamické programování}
Definujme funkci zisk, kterou chceme maximalizovat pro posloupnost délky N
$$zisk(N) = nejvy\check{s}\check{s}\acute{\i}\ suma\ v\ cel\acute{e}\  
posloupnosti \\ $$

\setlength{\unitlength}{.6mm}
\begin{picture}(150, 15)(-30, 0)

\put(-30, 3.5){\em N:}
\thinlines
\put(0, 0){\line(1, 0){150}}
\put(0, 10){\line(1, 0){150}}
\put(0, 0){\line(0, 1){10}}
\put(150, 0){\line(0, 1){10}}

\linethickness{.5mm}
% zisk(N)
\put(35, 3.5){\em zisk(N)}
\put(20, 0){\line(1, 0){50}}
\put(20, 10){\line(1, 0){50}}
\put(20, 0){\line(0, 1){10}}
\put(70, 0){\line(0, 1){10}}

\end{picture}
 \\

\vspace{1cm}

Abychom našli algoritmus založený na dynamickém programování, hledáme rekurzivní
vyjádření této funkce. Dodefinujeme ještě pomocné funkce

\begin{align*}
krajni\_zisk(N) &= nejvy\check{s}\check{s}\acute{\i}\ zisk\ z\ 
posloupnosti\ u\ prav\acute{e}ho\ kraje \\
h(N) &= hodnota\ N-t\acute{e}ho\ prvku
\end{align*}
\mbox{}

\begin{picture}(150, 15)(-30, 0)

%% figure 1 %%
\put(-30, 3.5){\em N:}

\thinlines
\put(0, 0){\line(1, 0){150}}
\put(0, 10){\line(1, 0){150}}
\put(0, 0){\line(0, 1){10}}
\put(150, 0){\line(0, 1){10}}

\linethickness{.5mm}
% zisk(N)
\put(35, 3.5){\em zisk(N)}
\put(20, 0){\line(1, 0){50}}
\put(20, 10){\line(1, 0){50}}
\put(20, 0){\line(0, 1){10}}
\put(70, 0){\line(0, 1){10}}

% krajni_max(N)
\put(96, 3.5){\em krajni\_zisk(N)}
\put(80, 0){\line(1, 0){70}}
\put(80, 10){\line(1, 0){70}}
\put(80, 0){\line(0, 1){10}}
\put(150, 0){\line(0, 1){10}}

\end{picture}
\mbox{}

\begin{picture}(150, 15)(-30, 0)

%% figure 2 %%
\put(-30, 3.5){\em N + 1:}
\thinlines
\put(0, 0){\line(1, 0){150}}
\put(0, 10){\line(1, 0){150}}
\put(0, 0){\line(0, 1){10}}
\put(150, 0){\line(0, 1){10}}

\linethickness{.5mm}
% zisk(N)
\put(35, 3.5){\em zisk(N)}
\put(20, 0){\line(1, 0){50}}
\put(20, 10){\line(1, 0){50}}
\put(20, 0){\line(0, 1){10}}
\put(70, 0){\line(0, 1){10}}

% krajni_max(N)
\put(96, 3.5){\em krajni\_zisk(N)}
\put(80, 0){\line(1, 0){70}}
\put(80, 10){\line(1, 0){70}}
\put(80, 0){\line(0, 1){10}}
\put(150, 0){\line(0, 1){10}}

% N + 1 hodnota
\put(155, 3.5){\em h(N + 1)}
\put(150, 0){\line(1, 0){35}}
\put(150, 10){\line(1, 0){35}}
\put(185, 0){\line(0, 1){10}}

\end{picture}


\mbox{}
\vspace{1cm}

\clearpage
Vyjádříme-li $zisk(N + 1)$ pomocí $zisk(N)$, mohou nastat tři případy.
\begin{enumerate}
\item krajní zisk nám vůbec nepomohl a $zisk(N + 1) = zisk(N)$ \\
\documentclass[11pt]{article}
\usepackage[utf8]{inputenc}
\usepackage[czech]{babel}
\usepackage{a4wide}
\usepackage{amsmath}


\include{pygments}


\title{Programovací strategie:  Maximalizujte svůj zisk}
\author{Tomáš Maršálek}
\date{6.\,října 2011}

%% -----------------------------------------


\begin{document}
\maketitle

\section{Zadání}
Se soutěžemi se v televizi roztrhl pytel. Je na čase, abyste si vydělali také 
vy. Otázky jsou pro vás naprosto triviální a přichází čas vybrat si odměnu. 
Přivedou před vás n asistentek. Asistentky se drží za ruce a každá má na krku 
pověšenou ceduli s celočíselnou sumou, kterou získáte (kladné číslo) nebo 
naopak ztratíte (záporné číslo), když si tuto asistentku vyberete. Můžete si 
vybrat jakýkoliv počet asistentek, ale jejich spojené ruce smíte přerušit max. 
na dvou místech, tedy posloupnost odměn, kterou si vyberete, musí být spojitá, 
ale může být libovolně dlouhá. Asistentky nezískáváte, pouze sumy výher, 
takže jiná než finanční kritéria není třeba brát v úvahu.

\section{Naivní řešení}
Jednoduše vyzkoušíme všechny souvislé podposloupnosti a vybereme tu s nejvyšší
sumou. Tedy pro všechny možné délky podposloupnosti a pro všechny jejich možné 
posuny zkoušíme, jestli je jejich suma lepší než dosavadní.
\begin{Verbatim}[commandchars=\\\{\}]
\PY{k}{def} \PY{n+nf}{naivniZisk}\PY{p}{(}\PY{n}{h}\PY{p}{)}\PY{p}{:}
	\PY{n}{n} \PY{o}{=} \PY{n+nb}{len}\PY{p}{(}\PY{n}{h}\PY{p}{)}
	\PY{n}{max\PYZus{}zisk}\PY{p}{,} \PY{n}{max\PYZus{}i}\PY{p}{,} \PY{n}{max\PYZus{}delka} \PY{o}{=} \PY{l+m+mi}{0}\PY{p}{,} \PY{l+m+mi}{0}\PY{p}{,} \PY{l+m+mi}{0}
	\PY{n}{pocatecni\PYZus{}zisk} \PY{o}{=} \PY{l+m+mi}{0}
	\PY{k}{for} \PY{n}{delka} \PY{o+ow}{in} \PY{n+nb}{range}\PY{p}{(}\PY{l+m+mi}{1}\PY{p}{,} \PY{n}{n} \PY{o}{+} \PY{l+m+mi}{1}\PY{p}{)}\PY{p}{:}
		\PY{n}{pocatecni\PYZus{}zisk} \PY{o}{+}\PY{o}{=} \PY{n}{h}\PY{p}{[}\PY{n}{delka} \PY{o}{-} \PY{l+m+mi}{1}\PY{p}{]}
		\PY{n}{zisk} \PY{o}{=} \PY{n}{pocatecni\PYZus{}zisk}

		\PY{k}{for} \PY{n}{i} \PY{o+ow}{in} \PY{n+nb}{range}\PY{p}{(}\PY{n}{n} \PY{o}{-} \PY{n}{delka}\PY{p}{)}\PY{p}{:}
			\PY{k}{if} \PY{n}{zisk} \PY{o}{>} \PY{n}{max\PYZus{}zisk}\PY{p}{:}
				\PY{n}{max\PYZus{}zisk}\PY{p}{,} \PY{n}{max\PYZus{}i}\PY{p}{,} \PY{n}{max\PYZus{}delka} \PY{o}{=} \PY{n}{zisk}\PY{p}{,} \PY{n}{i}\PY{p}{,} \PY{n}{delka}
			\PY{n}{zisk} \PY{o}{+}\PY{o}{=} \PY{n}{h}\PY{p}{[}\PY{n}{delka} \PY{o}{+} \PY{n}{i}\PY{p}{]} \PY{o}{-} \PY{n}{h}\PY{p}{[}\PY{n}{i}\PY{p}{]}

		\PY{k}{if} \PY{n}{zisk} \PY{o}{>} \PY{n}{max\PYZus{}zisk}\PY{p}{:}
			\PY{n}{max\PYZus{}zisk}\PY{p}{,} \PY{n}{max\PYZus{}i}\PY{p}{,} \PY{n}{max\PYZus{}delka} \PY{o}{=} \PY{n}{zisk}\PY{p}{,} \PY{n}{n} \PY{o}{-} \PY{n}{delka}\PY{p}{,} \PY{n}{delka}
	\PY{k}{return} \PY{n}{max\PYZus{}zisk}\PY{p}{,} \PY{n}{max\PYZus{}i}\PY{p}{,} \PY{n}{max\PYZus{}delka}
\end{Verbatim}


\subsection{Složitost}
Máme dvojitě vnořený cyklus, očekáváme kvadratickou složitost.
\begin{align*}
T(N) &= (\Theta(1) + (N - 1)) + (\Theta(1) + (N - 2)) + \ldots + (\Theta(1) + (N - N)) \\
&= \Theta(N) + N^2 - \frac{N}{2}(N + 1) \\
&= \Theta (N^2)
\end{align*}

\vspace{1cm}

\section{Dynamické programování}
Definujme funkci zisk, kterou chceme maximalizovat pro posloupnost délky N
$$zisk(N) = nejvy\check{s}\check{s}\acute{\i}\ suma\ v\ cel\acute{e}\  
posloupnosti \\ $$

\setlength{\unitlength}{.6mm}
\begin{picture}(150, 15)(-30, 0)

\put(-30, 3.5){\em N:}
\thinlines
\put(0, 0){\line(1, 0){150}}
\put(0, 10){\line(1, 0){150}}
\put(0, 0){\line(0, 1){10}}
\put(150, 0){\line(0, 1){10}}

\linethickness{.5mm}
% zisk(N)
\put(35, 3.5){\em zisk(N)}
\put(20, 0){\line(1, 0){50}}
\put(20, 10){\line(1, 0){50}}
\put(20, 0){\line(0, 1){10}}
\put(70, 0){\line(0, 1){10}}

\end{picture}
 \\

\vspace{1cm}

Abychom našli algoritmus založený na dynamickém programování, hledáme rekurzivní
vyjádření této funkce. Dodefinujeme ještě pomocné funkce

\begin{align*}
krajni\_zisk(N) &= nejvy\check{s}\check{s}\acute{\i}\ zisk\ z\ 
posloupnosti\ u\ prav\acute{e}ho\ kraje \\
h(N) &= hodnota\ N-t\acute{e}ho\ prvku
\end{align*}
\mbox{}

\begin{picture}(150, 15)(-30, 0)

%% figure 1 %%
\put(-30, 3.5){\em N:}

\thinlines
\put(0, 0){\line(1, 0){150}}
\put(0, 10){\line(1, 0){150}}
\put(0, 0){\line(0, 1){10}}
\put(150, 0){\line(0, 1){10}}

\linethickness{.5mm}
% zisk(N)
\put(35, 3.5){\em zisk(N)}
\put(20, 0){\line(1, 0){50}}
\put(20, 10){\line(1, 0){50}}
\put(20, 0){\line(0, 1){10}}
\put(70, 0){\line(0, 1){10}}

% krajni_max(N)
\put(96, 3.5){\em krajni\_zisk(N)}
\put(80, 0){\line(1, 0){70}}
\put(80, 10){\line(1, 0){70}}
\put(80, 0){\line(0, 1){10}}
\put(150, 0){\line(0, 1){10}}

\end{picture}
\mbox{}

\begin{picture}(150, 15)(-30, 0)

%% figure 2 %%
\put(-30, 3.5){\em N + 1:}
\thinlines
\put(0, 0){\line(1, 0){150}}
\put(0, 10){\line(1, 0){150}}
\put(0, 0){\line(0, 1){10}}
\put(150, 0){\line(0, 1){10}}

\linethickness{.5mm}
% zisk(N)
\put(35, 3.5){\em zisk(N)}
\put(20, 0){\line(1, 0){50}}
\put(20, 10){\line(1, 0){50}}
\put(20, 0){\line(0, 1){10}}
\put(70, 0){\line(0, 1){10}}

% krajni_max(N)
\put(96, 3.5){\em krajni\_zisk(N)}
\put(80, 0){\line(1, 0){70}}
\put(80, 10){\line(1, 0){70}}
\put(80, 0){\line(0, 1){10}}
\put(150, 0){\line(0, 1){10}}

% N + 1 hodnota
\put(155, 3.5){\em h(N + 1)}
\put(150, 0){\line(1, 0){35}}
\put(150, 10){\line(1, 0){35}}
\put(185, 0){\line(0, 1){10}}

\end{picture}


\mbox{}
\vspace{1cm}

\clearpage
Vyjádříme-li $zisk(N + 1)$ pomocí $zisk(N)$, mohou nastat tři případy.
\begin{enumerate}
\item krajní zisk nám vůbec nepomohl a $zisk(N + 1) = zisk(N)$ \\
\documentclass[11pt]{article}
\usepackage[utf8]{inputenc}
\usepackage[czech]{babel}
\usepackage{a4wide}
\usepackage{amsmath}


\include{pygments}


\title{Programovací strategie:  Maximalizujte svůj zisk}
\author{Tomáš Maršálek}
\date{6.\,října 2011}

%% -----------------------------------------


\begin{document}
\maketitle

\section{Zadání}
Se soutěžemi se v televizi roztrhl pytel. Je na čase, abyste si vydělali také 
vy. Otázky jsou pro vás naprosto triviální a přichází čas vybrat si odměnu. 
Přivedou před vás n asistentek. Asistentky se drží za ruce a každá má na krku 
pověšenou ceduli s celočíselnou sumou, kterou získáte (kladné číslo) nebo 
naopak ztratíte (záporné číslo), když si tuto asistentku vyberete. Můžete si 
vybrat jakýkoliv počet asistentek, ale jejich spojené ruce smíte přerušit max. 
na dvou místech, tedy posloupnost odměn, kterou si vyberete, musí být spojitá, 
ale může být libovolně dlouhá. Asistentky nezískáváte, pouze sumy výher, 
takže jiná než finanční kritéria není třeba brát v úvahu.

\section{Naivní řešení}
Jednoduše vyzkoušíme všechny souvislé podposloupnosti a vybereme tu s nejvyšší
sumou. Tedy pro všechny možné délky podposloupnosti a pro všechny jejich možné 
posuny zkoušíme, jestli je jejich suma lepší než dosavadní.
\input{naivniZisk.py}

\subsection{Složitost}
Máme dvojitě vnořený cyklus, očekáváme kvadratickou složitost.
\begin{align*}
T(N) &= (\Theta(1) + (N - 1)) + (\Theta(1) + (N - 2)) + \ldots + (\Theta(1) + (N - N)) \\
&= \Theta(N) + N^2 - \frac{N}{2}(N + 1) \\
&= \Theta (N^2)
\end{align*}

\vspace{1cm}

\section{Dynamické programování}
Definujme funkci zisk, kterou chceme maximalizovat pro posloupnost délky N
$$zisk(N) = nejvy\check{s}\check{s}\acute{\i}\ suma\ v\ cel\acute{e}\  
posloupnosti \\ $$

\setlength{\unitlength}{.6mm}
\input{zisk.fig1} \\

\vspace{1cm}

Abychom našli algoritmus založený na dynamickém programování, hledáme rekurzivní
vyjádření této funkce. Dodefinujeme ještě pomocné funkce

\begin{align*}
krajni\_zisk(N) &= nejvy\check{s}\check{s}\acute{\i}\ zisk\ z\ 
posloupnosti\ u\ prav\acute{e}ho\ kraje \\
h(N) &= hodnota\ N-t\acute{e}ho\ prvku
\end{align*}
\input{zisk.fig2}

\mbox{}
\vspace{1cm}

\clearpage
Vyjádříme-li $zisk(N + 1)$ pomocí $zisk(N)$, mohou nastat tři případy.
\begin{enumerate}
\item krajní zisk nám vůbec nepomohl a $zisk(N + 1) = zisk(N)$ \\
\input{zisk.fig3} \\

\item krajní zisk s novou hodnotou je větší než $zisk(N)$, tedy \\
	$zisk(N + 1) = krajni\_zisk(N) + h(N + 1)$ \\
\input{zisk.fig4} \\

\item $krajni\_zisk(N)$ je záporný, tedy je výhodnější vzít pouze 
	poslední hodnotu. \\
	$zisk(N + 1) = h(N + 1)$ \\
\input{zisk.fig5} \\

\end{enumerate} 

\vspace{1cm}

Rekurzivním vyjádřením pak dostáváme základ pro algoritmus. Rekurzi pouze 
obrátíme na dynamické programování.
\begin{align*}
zisk(N + 1) = \max (&zisk(N), \\
&krajni\_zisk(N) + h(N + 1), \\
&h(N + 1)) \\
\end{align*}

Stejně jako u předešlého algoritmu si zaznamenáváme hodnotu zisku, a polohu
podposloupnosti (počáteční index a délku). Náročnost je očividně $\Theta(N)$.
\input{dpZisk.py}

\section{Implementace}
Program {\tt zisk.py} je v jazyce python 2.7.2. Vstupní data jsou kladná 
a záporná celá čísla. Výstupem je hodnota zisku, index první asistentky 
(od nuly), která začíná vybranou posloupnost a délka této posloupnosti. Jsou 
implementována řešení oběma metodami.

\subsection{Příklad použití}
\begin{verbatim}
# vygenerovani vstupnich dat:  300 cisel od -500 do 499
> test.data
for i in $(seq 1 300); do echo $(($RANDOM % 1000 - 500)) >> test.data; done

cat test.data | ./zisk.py
\end{verbatim}

\end{document}
 \\

\item krajní zisk s novou hodnotou je větší než $zisk(N)$, tedy \\
	$zisk(N + 1) = krajni\_zisk(N) + h(N + 1)$ \\
\documentclass[11pt]{article}
\usepackage[utf8]{inputenc}
\usepackage[czech]{babel}
\usepackage{a4wide}
\usepackage{amsmath}


\include{pygments}


\title{Programovací strategie:  Maximalizujte svůj zisk}
\author{Tomáš Maršálek}
\date{6.\,října 2011}

%% -----------------------------------------


\begin{document}
\maketitle

\section{Zadání}
Se soutěžemi se v televizi roztrhl pytel. Je na čase, abyste si vydělali také 
vy. Otázky jsou pro vás naprosto triviální a přichází čas vybrat si odměnu. 
Přivedou před vás n asistentek. Asistentky se drží za ruce a každá má na krku 
pověšenou ceduli s celočíselnou sumou, kterou získáte (kladné číslo) nebo 
naopak ztratíte (záporné číslo), když si tuto asistentku vyberete. Můžete si 
vybrat jakýkoliv počet asistentek, ale jejich spojené ruce smíte přerušit max. 
na dvou místech, tedy posloupnost odměn, kterou si vyberete, musí být spojitá, 
ale může být libovolně dlouhá. Asistentky nezískáváte, pouze sumy výher, 
takže jiná než finanční kritéria není třeba brát v úvahu.

\section{Naivní řešení}
Jednoduše vyzkoušíme všechny souvislé podposloupnosti a vybereme tu s nejvyšší
sumou. Tedy pro všechny možné délky podposloupnosti a pro všechny jejich možné 
posuny zkoušíme, jestli je jejich suma lepší než dosavadní.
\input{naivniZisk.py}

\subsection{Složitost}
Máme dvojitě vnořený cyklus, očekáváme kvadratickou složitost.
\begin{align*}
T(N) &= (\Theta(1) + (N - 1)) + (\Theta(1) + (N - 2)) + \ldots + (\Theta(1) + (N - N)) \\
&= \Theta(N) + N^2 - \frac{N}{2}(N + 1) \\
&= \Theta (N^2)
\end{align*}

\vspace{1cm}

\section{Dynamické programování}
Definujme funkci zisk, kterou chceme maximalizovat pro posloupnost délky N
$$zisk(N) = nejvy\check{s}\check{s}\acute{\i}\ suma\ v\ cel\acute{e}\  
posloupnosti \\ $$

\setlength{\unitlength}{.6mm}
\input{zisk.fig1} \\

\vspace{1cm}

Abychom našli algoritmus založený na dynamickém programování, hledáme rekurzivní
vyjádření této funkce. Dodefinujeme ještě pomocné funkce

\begin{align*}
krajni\_zisk(N) &= nejvy\check{s}\check{s}\acute{\i}\ zisk\ z\ 
posloupnosti\ u\ prav\acute{e}ho\ kraje \\
h(N) &= hodnota\ N-t\acute{e}ho\ prvku
\end{align*}
\input{zisk.fig2}

\mbox{}
\vspace{1cm}

\clearpage
Vyjádříme-li $zisk(N + 1)$ pomocí $zisk(N)$, mohou nastat tři případy.
\begin{enumerate}
\item krajní zisk nám vůbec nepomohl a $zisk(N + 1) = zisk(N)$ \\
\input{zisk.fig3} \\

\item krajní zisk s novou hodnotou je větší než $zisk(N)$, tedy \\
	$zisk(N + 1) = krajni\_zisk(N) + h(N + 1)$ \\
\input{zisk.fig4} \\

\item $krajni\_zisk(N)$ je záporný, tedy je výhodnější vzít pouze 
	poslední hodnotu. \\
	$zisk(N + 1) = h(N + 1)$ \\
\input{zisk.fig5} \\

\end{enumerate} 

\vspace{1cm}

Rekurzivním vyjádřením pak dostáváme základ pro algoritmus. Rekurzi pouze 
obrátíme na dynamické programování.
\begin{align*}
zisk(N + 1) = \max (&zisk(N), \\
&krajni\_zisk(N) + h(N + 1), \\
&h(N + 1)) \\
\end{align*}

Stejně jako u předešlého algoritmu si zaznamenáváme hodnotu zisku, a polohu
podposloupnosti (počáteční index a délku). Náročnost je očividně $\Theta(N)$.
\input{dpZisk.py}

\section{Implementace}
Program {\tt zisk.py} je v jazyce python 2.7.2. Vstupní data jsou kladná 
a záporná celá čísla. Výstupem je hodnota zisku, index první asistentky 
(od nuly), která začíná vybranou posloupnost a délka této posloupnosti. Jsou 
implementována řešení oběma metodami.

\subsection{Příklad použití}
\begin{verbatim}
# vygenerovani vstupnich dat:  300 cisel od -500 do 499
> test.data
for i in $(seq 1 300); do echo $(($RANDOM % 1000 - 500)) >> test.data; done

cat test.data | ./zisk.py
\end{verbatim}

\end{document}
 \\

\item $krajni\_zisk(N)$ je záporný, tedy je výhodnější vzít pouze 
	poslední hodnotu. \\
	$zisk(N + 1) = h(N + 1)$ \\
\begin{picture}(150, 15)(-30, 0)

\put(-30, 3.5){\em N + 1:}
\thinlines
\put(0, 0){\line(1, 0){185}}
\put(0, 10){\line(1, 0){185}}
\put(0, 0){\line(0, 1){10}}
\put(185, 0){\line(0, 1){10}}

\linethickness{.5mm}

% N + 1 hodnota
\put(155, 3.5){\em h(N + 1)}
\put(150, 0){\line(1, 0){35}}
\put(150, 10){\line(1, 0){35}}
\put(185, 0){\line(0, 1){10}}
\put(150, 0){\line(0, 1){10}}

\end{picture}
 \\

\end{enumerate} 

\vspace{1cm}

Rekurzivním vyjádřením pak dostáváme základ pro algoritmus. Rekurzi pouze 
obrátíme na dynamické programování.
\begin{align*}
zisk(N + 1) = \max (&zisk(N), \\
&krajni\_zisk(N) + h(N + 1), \\
&h(N + 1)) \\
\end{align*}

Stejně jako u předešlého algoritmu si zaznamenáváme hodnotu zisku, a polohu
podposloupnosti (počáteční index a délku). Náročnost je očividně $\Theta(N)$.
\begin{Verbatim}[commandchars=\\\{\}]
\PY{k}{def} \PY{n+nf}{dpZisk}\PY{p}{(}\PY{n}{h}\PY{p}{)}\PY{p}{:}
	\PY{n}{n} \PY{o}{=} \PY{n+nb}{len}\PY{p}{(}\PY{n}{h}\PY{p}{)}
	\PY{n}{max\PYZus{}zisk}\PY{p}{,} \PY{n}{max\PYZus{}i}\PY{p}{,} \PY{n}{max\PYZus{}delka} \PY{o}{=} \PY{l+m+mi}{0}\PY{p}{,} \PY{l+m+mi}{0}\PY{p}{,} \PY{l+m+mi}{0}
	\PY{n}{krajni\PYZus{}zisk}\PY{p}{,} \PY{n}{krajni\PYZus{}i}\PY{p}{,} \PY{n}{krajni\PYZus{}delka} \PY{o}{=} \PY{l+m+mi}{0}\PY{p}{,} \PY{l+m+mi}{0}\PY{p}{,} \PY{l+m+mi}{0}
	\PY{k}{for} \PY{n}{i} \PY{o+ow}{in} \PY{n+nb}{range}\PY{p}{(}\PY{n}{n}\PY{p}{)}\PY{p}{:}
		\PY{k}{if} \PY{n}{krajni\PYZus{}zisk} \PY{o}{>} \PY{l+m+mi}{0}\PY{p}{:}        \PY{c}{\PYZsh{} Případ 2}
			\PY{n}{krajni\PYZus{}zisk} \PY{o}{+}\PY{o}{=} \PY{n}{h}\PY{p}{[}\PY{n}{i}\PY{p}{]}
			\PY{n}{krajni\PYZus{}delka} \PY{o}{+}\PY{o}{=} \PY{l+m+mi}{1}
		\PY{k}{else}\PY{p}{:}                      \PY{c}{\PYZsh{} Případ 3}
			\PY{n}{krajni\PYZus{}zisk} \PY{p}{,}\PY{n}{krajni\PYZus{}i}\PY{p}{,} \PY{n}{krajni\PYZus{}delka} \PY{o}{=} \PY{n}{h}\PY{p}{[}\PY{n}{i}\PY{p}{]}\PY{p}{,} \PY{n}{i}\PY{p}{,} \PY{l+m+mi}{1}

		\PY{k}{if} \PY{n}{krajni\PYZus{}zisk} \PY{o}{>} \PY{n}{max\PYZus{}zisk}\PY{p}{:} \PY{c}{\PYZsh{} Případ 1 (obráceně)}
			\PY{n}{max\PYZus{}zisk}\PY{p}{,} \PY{n}{max\PYZus{}i}\PY{p}{,} \PY{n}{max\PYZus{}delka} \PY{o}{=} \PY{n}{krajni\PYZus{}zisk}\PY{p}{,} \PY{n}{krajni\PYZus{}i}\PY{p}{,} \PY{n}{krajni\PYZus{}delka}
	\PY{k}{return} \PY{n}{max\PYZus{}zisk}\PY{p}{,} \PY{n}{max\PYZus{}i}\PY{p}{,} \PY{n}{max\PYZus{}delka}
\end{Verbatim}


\section{Implementace}
Program {\tt zisk.py} je v jazyce python 2.7.2. Vstupní data jsou kladná 
a záporná celá čísla. Výstupem je hodnota zisku, index první asistentky 
(od nuly), která začíná vybranou posloupnost a délka této posloupnosti. Jsou 
implementována řešení oběma metodami.

\subsection{Příklad použití}
\begin{verbatim}
# vygenerovani vstupnich dat:  300 cisel od -500 do 499
> test.data
for i in $(seq 1 300); do echo $(($RANDOM % 1000 - 500)) >> test.data; done

cat test.data | ./zisk.py
\end{verbatim}

\end{document}
 \\

\item krajní zisk s novou hodnotou je větší než $zisk(N)$, tedy \\
	$zisk(N + 1) = krajni\_zisk(N) + h(N + 1)$ \\
\documentclass[11pt]{article}
\usepackage[utf8]{inputenc}
\usepackage[czech]{babel}
\usepackage{a4wide}
\usepackage{amsmath}


\include{pygments}


\title{Programovací strategie:  Maximalizujte svůj zisk}
\author{Tomáš Maršálek}
\date{6.\,října 2011}

%% -----------------------------------------


\begin{document}
\maketitle

\section{Zadání}
Se soutěžemi se v televizi roztrhl pytel. Je na čase, abyste si vydělali také 
vy. Otázky jsou pro vás naprosto triviální a přichází čas vybrat si odměnu. 
Přivedou před vás n asistentek. Asistentky se drží za ruce a každá má na krku 
pověšenou ceduli s celočíselnou sumou, kterou získáte (kladné číslo) nebo 
naopak ztratíte (záporné číslo), když si tuto asistentku vyberete. Můžete si 
vybrat jakýkoliv počet asistentek, ale jejich spojené ruce smíte přerušit max. 
na dvou místech, tedy posloupnost odměn, kterou si vyberete, musí být spojitá, 
ale může být libovolně dlouhá. Asistentky nezískáváte, pouze sumy výher, 
takže jiná než finanční kritéria není třeba brát v úvahu.

\section{Naivní řešení}
Jednoduše vyzkoušíme všechny souvislé podposloupnosti a vybereme tu s nejvyšší
sumou. Tedy pro všechny možné délky podposloupnosti a pro všechny jejich možné 
posuny zkoušíme, jestli je jejich suma lepší než dosavadní.
\begin{Verbatim}[commandchars=\\\{\}]
\PY{k}{def} \PY{n+nf}{naivniZisk}\PY{p}{(}\PY{n}{h}\PY{p}{)}\PY{p}{:}
	\PY{n}{n} \PY{o}{=} \PY{n+nb}{len}\PY{p}{(}\PY{n}{h}\PY{p}{)}
	\PY{n}{max\PYZus{}zisk}\PY{p}{,} \PY{n}{max\PYZus{}i}\PY{p}{,} \PY{n}{max\PYZus{}delka} \PY{o}{=} \PY{l+m+mi}{0}\PY{p}{,} \PY{l+m+mi}{0}\PY{p}{,} \PY{l+m+mi}{0}
	\PY{n}{pocatecni\PYZus{}zisk} \PY{o}{=} \PY{l+m+mi}{0}
	\PY{k}{for} \PY{n}{delka} \PY{o+ow}{in} \PY{n+nb}{range}\PY{p}{(}\PY{l+m+mi}{1}\PY{p}{,} \PY{n}{n} \PY{o}{+} \PY{l+m+mi}{1}\PY{p}{)}\PY{p}{:}
		\PY{n}{pocatecni\PYZus{}zisk} \PY{o}{+}\PY{o}{=} \PY{n}{h}\PY{p}{[}\PY{n}{delka} \PY{o}{-} \PY{l+m+mi}{1}\PY{p}{]}
		\PY{n}{zisk} \PY{o}{=} \PY{n}{pocatecni\PYZus{}zisk}

		\PY{k}{for} \PY{n}{i} \PY{o+ow}{in} \PY{n+nb}{range}\PY{p}{(}\PY{n}{n} \PY{o}{-} \PY{n}{delka}\PY{p}{)}\PY{p}{:}
			\PY{k}{if} \PY{n}{zisk} \PY{o}{>} \PY{n}{max\PYZus{}zisk}\PY{p}{:}
				\PY{n}{max\PYZus{}zisk}\PY{p}{,} \PY{n}{max\PYZus{}i}\PY{p}{,} \PY{n}{max\PYZus{}delka} \PY{o}{=} \PY{n}{zisk}\PY{p}{,} \PY{n}{i}\PY{p}{,} \PY{n}{delka}
			\PY{n}{zisk} \PY{o}{+}\PY{o}{=} \PY{n}{h}\PY{p}{[}\PY{n}{delka} \PY{o}{+} \PY{n}{i}\PY{p}{]} \PY{o}{-} \PY{n}{h}\PY{p}{[}\PY{n}{i}\PY{p}{]}

		\PY{k}{if} \PY{n}{zisk} \PY{o}{>} \PY{n}{max\PYZus{}zisk}\PY{p}{:}
			\PY{n}{max\PYZus{}zisk}\PY{p}{,} \PY{n}{max\PYZus{}i}\PY{p}{,} \PY{n}{max\PYZus{}delka} \PY{o}{=} \PY{n}{zisk}\PY{p}{,} \PY{n}{n} \PY{o}{-} \PY{n}{delka}\PY{p}{,} \PY{n}{delka}
	\PY{k}{return} \PY{n}{max\PYZus{}zisk}\PY{p}{,} \PY{n}{max\PYZus{}i}\PY{p}{,} \PY{n}{max\PYZus{}delka}
\end{Verbatim}


\subsection{Složitost}
Máme dvojitě vnořený cyklus, očekáváme kvadratickou složitost.
\begin{align*}
T(N) &= (\Theta(1) + (N - 1)) + (\Theta(1) + (N - 2)) + \ldots + (\Theta(1) + (N - N)) \\
&= \Theta(N) + N^2 - \frac{N}{2}(N + 1) \\
&= \Theta (N^2)
\end{align*}

\vspace{1cm}

\section{Dynamické programování}
Definujme funkci zisk, kterou chceme maximalizovat pro posloupnost délky N
$$zisk(N) = nejvy\check{s}\check{s}\acute{\i}\ suma\ v\ cel\acute{e}\  
posloupnosti \\ $$

\setlength{\unitlength}{.6mm}
\begin{picture}(150, 15)(-30, 0)

\put(-30, 3.5){\em N:}
\thinlines
\put(0, 0){\line(1, 0){150}}
\put(0, 10){\line(1, 0){150}}
\put(0, 0){\line(0, 1){10}}
\put(150, 0){\line(0, 1){10}}

\linethickness{.5mm}
% zisk(N)
\put(35, 3.5){\em zisk(N)}
\put(20, 0){\line(1, 0){50}}
\put(20, 10){\line(1, 0){50}}
\put(20, 0){\line(0, 1){10}}
\put(70, 0){\line(0, 1){10}}

\end{picture}
 \\

\vspace{1cm}

Abychom našli algoritmus založený na dynamickém programování, hledáme rekurzivní
vyjádření této funkce. Dodefinujeme ještě pomocné funkce

\begin{align*}
krajni\_zisk(N) &= nejvy\check{s}\check{s}\acute{\i}\ zisk\ z\ 
posloupnosti\ u\ prav\acute{e}ho\ kraje \\
h(N) &= hodnota\ N-t\acute{e}ho\ prvku
\end{align*}
\mbox{}

\begin{picture}(150, 15)(-30, 0)

%% figure 1 %%
\put(-30, 3.5){\em N:}

\thinlines
\put(0, 0){\line(1, 0){150}}
\put(0, 10){\line(1, 0){150}}
\put(0, 0){\line(0, 1){10}}
\put(150, 0){\line(0, 1){10}}

\linethickness{.5mm}
% zisk(N)
\put(35, 3.5){\em zisk(N)}
\put(20, 0){\line(1, 0){50}}
\put(20, 10){\line(1, 0){50}}
\put(20, 0){\line(0, 1){10}}
\put(70, 0){\line(0, 1){10}}

% krajni_max(N)
\put(96, 3.5){\em krajni\_zisk(N)}
\put(80, 0){\line(1, 0){70}}
\put(80, 10){\line(1, 0){70}}
\put(80, 0){\line(0, 1){10}}
\put(150, 0){\line(0, 1){10}}

\end{picture}
\mbox{}

\begin{picture}(150, 15)(-30, 0)

%% figure 2 %%
\put(-30, 3.5){\em N + 1:}
\thinlines
\put(0, 0){\line(1, 0){150}}
\put(0, 10){\line(1, 0){150}}
\put(0, 0){\line(0, 1){10}}
\put(150, 0){\line(0, 1){10}}

\linethickness{.5mm}
% zisk(N)
\put(35, 3.5){\em zisk(N)}
\put(20, 0){\line(1, 0){50}}
\put(20, 10){\line(1, 0){50}}
\put(20, 0){\line(0, 1){10}}
\put(70, 0){\line(0, 1){10}}

% krajni_max(N)
\put(96, 3.5){\em krajni\_zisk(N)}
\put(80, 0){\line(1, 0){70}}
\put(80, 10){\line(1, 0){70}}
\put(80, 0){\line(0, 1){10}}
\put(150, 0){\line(0, 1){10}}

% N + 1 hodnota
\put(155, 3.5){\em h(N + 1)}
\put(150, 0){\line(1, 0){35}}
\put(150, 10){\line(1, 0){35}}
\put(185, 0){\line(0, 1){10}}

\end{picture}


\mbox{}
\vspace{1cm}

\clearpage
Vyjádříme-li $zisk(N + 1)$ pomocí $zisk(N)$, mohou nastat tři případy.
\begin{enumerate}
\item krajní zisk nám vůbec nepomohl a $zisk(N + 1) = zisk(N)$ \\
\documentclass[11pt]{article}
\usepackage[utf8]{inputenc}
\usepackage[czech]{babel}
\usepackage{a4wide}
\usepackage{amsmath}


\include{pygments}


\title{Programovací strategie:  Maximalizujte svůj zisk}
\author{Tomáš Maršálek}
\date{6.\,října 2011}

%% -----------------------------------------


\begin{document}
\maketitle

\section{Zadání}
Se soutěžemi se v televizi roztrhl pytel. Je na čase, abyste si vydělali také 
vy. Otázky jsou pro vás naprosto triviální a přichází čas vybrat si odměnu. 
Přivedou před vás n asistentek. Asistentky se drží za ruce a každá má na krku 
pověšenou ceduli s celočíselnou sumou, kterou získáte (kladné číslo) nebo 
naopak ztratíte (záporné číslo), když si tuto asistentku vyberete. Můžete si 
vybrat jakýkoliv počet asistentek, ale jejich spojené ruce smíte přerušit max. 
na dvou místech, tedy posloupnost odměn, kterou si vyberete, musí být spojitá, 
ale může být libovolně dlouhá. Asistentky nezískáváte, pouze sumy výher, 
takže jiná než finanční kritéria není třeba brát v úvahu.

\section{Naivní řešení}
Jednoduše vyzkoušíme všechny souvislé podposloupnosti a vybereme tu s nejvyšší
sumou. Tedy pro všechny možné délky podposloupnosti a pro všechny jejich možné 
posuny zkoušíme, jestli je jejich suma lepší než dosavadní.
\input{naivniZisk.py}

\subsection{Složitost}
Máme dvojitě vnořený cyklus, očekáváme kvadratickou složitost.
\begin{align*}
T(N) &= (\Theta(1) + (N - 1)) + (\Theta(1) + (N - 2)) + \ldots + (\Theta(1) + (N - N)) \\
&= \Theta(N) + N^2 - \frac{N}{2}(N + 1) \\
&= \Theta (N^2)
\end{align*}

\vspace{1cm}

\section{Dynamické programování}
Definujme funkci zisk, kterou chceme maximalizovat pro posloupnost délky N
$$zisk(N) = nejvy\check{s}\check{s}\acute{\i}\ suma\ v\ cel\acute{e}\  
posloupnosti \\ $$

\setlength{\unitlength}{.6mm}
\input{zisk.fig1} \\

\vspace{1cm}

Abychom našli algoritmus založený na dynamickém programování, hledáme rekurzivní
vyjádření této funkce. Dodefinujeme ještě pomocné funkce

\begin{align*}
krajni\_zisk(N) &= nejvy\check{s}\check{s}\acute{\i}\ zisk\ z\ 
posloupnosti\ u\ prav\acute{e}ho\ kraje \\
h(N) &= hodnota\ N-t\acute{e}ho\ prvku
\end{align*}
\input{zisk.fig2}

\mbox{}
\vspace{1cm}

\clearpage
Vyjádříme-li $zisk(N + 1)$ pomocí $zisk(N)$, mohou nastat tři případy.
\begin{enumerate}
\item krajní zisk nám vůbec nepomohl a $zisk(N + 1) = zisk(N)$ \\
\input{zisk.fig3} \\

\item krajní zisk s novou hodnotou je větší než $zisk(N)$, tedy \\
	$zisk(N + 1) = krajni\_zisk(N) + h(N + 1)$ \\
\input{zisk.fig4} \\

\item $krajni\_zisk(N)$ je záporný, tedy je výhodnější vzít pouze 
	poslední hodnotu. \\
	$zisk(N + 1) = h(N + 1)$ \\
\input{zisk.fig5} \\

\end{enumerate} 

\vspace{1cm}

Rekurzivním vyjádřením pak dostáváme základ pro algoritmus. Rekurzi pouze 
obrátíme na dynamické programování.
\begin{align*}
zisk(N + 1) = \max (&zisk(N), \\
&krajni\_zisk(N) + h(N + 1), \\
&h(N + 1)) \\
\end{align*}

Stejně jako u předešlého algoritmu si zaznamenáváme hodnotu zisku, a polohu
podposloupnosti (počáteční index a délku). Náročnost je očividně $\Theta(N)$.
\input{dpZisk.py}

\section{Implementace}
Program {\tt zisk.py} je v jazyce python 2.7.2. Vstupní data jsou kladná 
a záporná celá čísla. Výstupem je hodnota zisku, index první asistentky 
(od nuly), která začíná vybranou posloupnost a délka této posloupnosti. Jsou 
implementována řešení oběma metodami.

\subsection{Příklad použití}
\begin{verbatim}
# vygenerovani vstupnich dat:  300 cisel od -500 do 499
> test.data
for i in $(seq 1 300); do echo $(($RANDOM % 1000 - 500)) >> test.data; done

cat test.data | ./zisk.py
\end{verbatim}

\end{document}
 \\

\item krajní zisk s novou hodnotou je větší než $zisk(N)$, tedy \\
	$zisk(N + 1) = krajni\_zisk(N) + h(N + 1)$ \\
\documentclass[11pt]{article}
\usepackage[utf8]{inputenc}
\usepackage[czech]{babel}
\usepackage{a4wide}
\usepackage{amsmath}


\include{pygments}


\title{Programovací strategie:  Maximalizujte svůj zisk}
\author{Tomáš Maršálek}
\date{6.\,října 2011}

%% -----------------------------------------


\begin{document}
\maketitle

\section{Zadání}
Se soutěžemi se v televizi roztrhl pytel. Je na čase, abyste si vydělali také 
vy. Otázky jsou pro vás naprosto triviální a přichází čas vybrat si odměnu. 
Přivedou před vás n asistentek. Asistentky se drží za ruce a každá má na krku 
pověšenou ceduli s celočíselnou sumou, kterou získáte (kladné číslo) nebo 
naopak ztratíte (záporné číslo), když si tuto asistentku vyberete. Můžete si 
vybrat jakýkoliv počet asistentek, ale jejich spojené ruce smíte přerušit max. 
na dvou místech, tedy posloupnost odměn, kterou si vyberete, musí být spojitá, 
ale může být libovolně dlouhá. Asistentky nezískáváte, pouze sumy výher, 
takže jiná než finanční kritéria není třeba brát v úvahu.

\section{Naivní řešení}
Jednoduše vyzkoušíme všechny souvislé podposloupnosti a vybereme tu s nejvyšší
sumou. Tedy pro všechny možné délky podposloupnosti a pro všechny jejich možné 
posuny zkoušíme, jestli je jejich suma lepší než dosavadní.
\input{naivniZisk.py}

\subsection{Složitost}
Máme dvojitě vnořený cyklus, očekáváme kvadratickou složitost.
\begin{align*}
T(N) &= (\Theta(1) + (N - 1)) + (\Theta(1) + (N - 2)) + \ldots + (\Theta(1) + (N - N)) \\
&= \Theta(N) + N^2 - \frac{N}{2}(N + 1) \\
&= \Theta (N^2)
\end{align*}

\vspace{1cm}

\section{Dynamické programování}
Definujme funkci zisk, kterou chceme maximalizovat pro posloupnost délky N
$$zisk(N) = nejvy\check{s}\check{s}\acute{\i}\ suma\ v\ cel\acute{e}\  
posloupnosti \\ $$

\setlength{\unitlength}{.6mm}
\input{zisk.fig1} \\

\vspace{1cm}

Abychom našli algoritmus založený na dynamickém programování, hledáme rekurzivní
vyjádření této funkce. Dodefinujeme ještě pomocné funkce

\begin{align*}
krajni\_zisk(N) &= nejvy\check{s}\check{s}\acute{\i}\ zisk\ z\ 
posloupnosti\ u\ prav\acute{e}ho\ kraje \\
h(N) &= hodnota\ N-t\acute{e}ho\ prvku
\end{align*}
\input{zisk.fig2}

\mbox{}
\vspace{1cm}

\clearpage
Vyjádříme-li $zisk(N + 1)$ pomocí $zisk(N)$, mohou nastat tři případy.
\begin{enumerate}
\item krajní zisk nám vůbec nepomohl a $zisk(N + 1) = zisk(N)$ \\
\input{zisk.fig3} \\

\item krajní zisk s novou hodnotou je větší než $zisk(N)$, tedy \\
	$zisk(N + 1) = krajni\_zisk(N) + h(N + 1)$ \\
\input{zisk.fig4} \\

\item $krajni\_zisk(N)$ je záporný, tedy je výhodnější vzít pouze 
	poslední hodnotu. \\
	$zisk(N + 1) = h(N + 1)$ \\
\input{zisk.fig5} \\

\end{enumerate} 

\vspace{1cm}

Rekurzivním vyjádřením pak dostáváme základ pro algoritmus. Rekurzi pouze 
obrátíme na dynamické programování.
\begin{align*}
zisk(N + 1) = \max (&zisk(N), \\
&krajni\_zisk(N) + h(N + 1), \\
&h(N + 1)) \\
\end{align*}

Stejně jako u předešlého algoritmu si zaznamenáváme hodnotu zisku, a polohu
podposloupnosti (počáteční index a délku). Náročnost je očividně $\Theta(N)$.
\input{dpZisk.py}

\section{Implementace}
Program {\tt zisk.py} je v jazyce python 2.7.2. Vstupní data jsou kladná 
a záporná celá čísla. Výstupem je hodnota zisku, index první asistentky 
(od nuly), která začíná vybranou posloupnost a délka této posloupnosti. Jsou 
implementována řešení oběma metodami.

\subsection{Příklad použití}
\begin{verbatim}
# vygenerovani vstupnich dat:  300 cisel od -500 do 499
> test.data
for i in $(seq 1 300); do echo $(($RANDOM % 1000 - 500)) >> test.data; done

cat test.data | ./zisk.py
\end{verbatim}

\end{document}
 \\

\item $krajni\_zisk(N)$ je záporný, tedy je výhodnější vzít pouze 
	poslední hodnotu. \\
	$zisk(N + 1) = h(N + 1)$ \\
\begin{picture}(150, 15)(-30, 0)

\put(-30, 3.5){\em N + 1:}
\thinlines
\put(0, 0){\line(1, 0){185}}
\put(0, 10){\line(1, 0){185}}
\put(0, 0){\line(0, 1){10}}
\put(185, 0){\line(0, 1){10}}

\linethickness{.5mm}

% N + 1 hodnota
\put(155, 3.5){\em h(N + 1)}
\put(150, 0){\line(1, 0){35}}
\put(150, 10){\line(1, 0){35}}
\put(185, 0){\line(0, 1){10}}
\put(150, 0){\line(0, 1){10}}

\end{picture}
 \\

\end{enumerate} 

\vspace{1cm}

Rekurzivním vyjádřením pak dostáváme základ pro algoritmus. Rekurzi pouze 
obrátíme na dynamické programování.
\begin{align*}
zisk(N + 1) = \max (&zisk(N), \\
&krajni\_zisk(N) + h(N + 1), \\
&h(N + 1)) \\
\end{align*}

Stejně jako u předešlého algoritmu si zaznamenáváme hodnotu zisku, a polohu
podposloupnosti (počáteční index a délku). Náročnost je očividně $\Theta(N)$.
\begin{Verbatim}[commandchars=\\\{\}]
\PY{k}{def} \PY{n+nf}{dpZisk}\PY{p}{(}\PY{n}{h}\PY{p}{)}\PY{p}{:}
	\PY{n}{n} \PY{o}{=} \PY{n+nb}{len}\PY{p}{(}\PY{n}{h}\PY{p}{)}
	\PY{n}{max\PYZus{}zisk}\PY{p}{,} \PY{n}{max\PYZus{}i}\PY{p}{,} \PY{n}{max\PYZus{}delka} \PY{o}{=} \PY{l+m+mi}{0}\PY{p}{,} \PY{l+m+mi}{0}\PY{p}{,} \PY{l+m+mi}{0}
	\PY{n}{krajni\PYZus{}zisk}\PY{p}{,} \PY{n}{krajni\PYZus{}i}\PY{p}{,} \PY{n}{krajni\PYZus{}delka} \PY{o}{=} \PY{l+m+mi}{0}\PY{p}{,} \PY{l+m+mi}{0}\PY{p}{,} \PY{l+m+mi}{0}
	\PY{k}{for} \PY{n}{i} \PY{o+ow}{in} \PY{n+nb}{range}\PY{p}{(}\PY{n}{n}\PY{p}{)}\PY{p}{:}
		\PY{k}{if} \PY{n}{krajni\PYZus{}zisk} \PY{o}{>} \PY{l+m+mi}{0}\PY{p}{:}        \PY{c}{\PYZsh{} Případ 2}
			\PY{n}{krajni\PYZus{}zisk} \PY{o}{+}\PY{o}{=} \PY{n}{h}\PY{p}{[}\PY{n}{i}\PY{p}{]}
			\PY{n}{krajni\PYZus{}delka} \PY{o}{+}\PY{o}{=} \PY{l+m+mi}{1}
		\PY{k}{else}\PY{p}{:}                      \PY{c}{\PYZsh{} Případ 3}
			\PY{n}{krajni\PYZus{}zisk} \PY{p}{,}\PY{n}{krajni\PYZus{}i}\PY{p}{,} \PY{n}{krajni\PYZus{}delka} \PY{o}{=} \PY{n}{h}\PY{p}{[}\PY{n}{i}\PY{p}{]}\PY{p}{,} \PY{n}{i}\PY{p}{,} \PY{l+m+mi}{1}

		\PY{k}{if} \PY{n}{krajni\PYZus{}zisk} \PY{o}{>} \PY{n}{max\PYZus{}zisk}\PY{p}{:} \PY{c}{\PYZsh{} Případ 1 (obráceně)}
			\PY{n}{max\PYZus{}zisk}\PY{p}{,} \PY{n}{max\PYZus{}i}\PY{p}{,} \PY{n}{max\PYZus{}delka} \PY{o}{=} \PY{n}{krajni\PYZus{}zisk}\PY{p}{,} \PY{n}{krajni\PYZus{}i}\PY{p}{,} \PY{n}{krajni\PYZus{}delka}
	\PY{k}{return} \PY{n}{max\PYZus{}zisk}\PY{p}{,} \PY{n}{max\PYZus{}i}\PY{p}{,} \PY{n}{max\PYZus{}delka}
\end{Verbatim}


\section{Implementace}
Program {\tt zisk.py} je v jazyce python 2.7.2. Vstupní data jsou kladná 
a záporná celá čísla. Výstupem je hodnota zisku, index první asistentky 
(od nuly), která začíná vybranou posloupnost a délka této posloupnosti. Jsou 
implementována řešení oběma metodami.

\subsection{Příklad použití}
\begin{verbatim}
# vygenerovani vstupnich dat:  300 cisel od -500 do 499
> test.data
for i in $(seq 1 300); do echo $(($RANDOM % 1000 - 500)) >> test.data; done

cat test.data | ./zisk.py
\end{verbatim}

\end{document}
 \\

\item $krajni\_zisk(N)$ je záporný, tedy je výhodnější vzít pouze 
	poslední hodnotu. \\
	$zisk(N + 1) = h(N + 1)$ \\
\begin{picture}(150, 15)(-30, 0)

\put(-30, 3.5){\em N + 1:}
\thinlines
\put(0, 0){\line(1, 0){185}}
\put(0, 10){\line(1, 0){185}}
\put(0, 0){\line(0, 1){10}}
\put(185, 0){\line(0, 1){10}}

\linethickness{.5mm}

% N + 1 hodnota
\put(155, 3.5){\em h(N + 1)}
\put(150, 0){\line(1, 0){35}}
\put(150, 10){\line(1, 0){35}}
\put(185, 0){\line(0, 1){10}}
\put(150, 0){\line(0, 1){10}}

\end{picture}
 \\

\end{enumerate} 

\vspace{1cm}

Rekurzivním vyjádřením pak dostáváme základ pro algoritmus. Rekurzi pouze 
obrátíme na dynamické programování.
\begin{align*}
zisk(N + 1) = \max (&zisk(N), \\
&krajni\_zisk(N) + h(N + 1), \\
&h(N + 1)) \\
\end{align*}

Stejně jako u předešlého algoritmu si zaznamenáváme hodnotu zisku, a polohu
podposloupnosti (počáteční index a délku). Náročnost je očividně $\Theta(N)$.
\begin{Verbatim}[commandchars=\\\{\}]
\PY{k}{def} \PY{n+nf}{dpZisk}\PY{p}{(}\PY{n}{h}\PY{p}{)}\PY{p}{:}
	\PY{n}{n} \PY{o}{=} \PY{n+nb}{len}\PY{p}{(}\PY{n}{h}\PY{p}{)}
	\PY{n}{max\PYZus{}zisk}\PY{p}{,} \PY{n}{max\PYZus{}i}\PY{p}{,} \PY{n}{max\PYZus{}delka} \PY{o}{=} \PY{l+m+mi}{0}\PY{p}{,} \PY{l+m+mi}{0}\PY{p}{,} \PY{l+m+mi}{0}
	\PY{n}{krajni\PYZus{}zisk}\PY{p}{,} \PY{n}{krajni\PYZus{}i}\PY{p}{,} \PY{n}{krajni\PYZus{}delka} \PY{o}{=} \PY{l+m+mi}{0}\PY{p}{,} \PY{l+m+mi}{0}\PY{p}{,} \PY{l+m+mi}{0}
	\PY{k}{for} \PY{n}{i} \PY{o+ow}{in} \PY{n+nb}{range}\PY{p}{(}\PY{n}{n}\PY{p}{)}\PY{p}{:}
		\PY{k}{if} \PY{n}{krajni\PYZus{}zisk} \PY{o}{>} \PY{l+m+mi}{0}\PY{p}{:}        \PY{c}{\PYZsh{} Případ 2}
			\PY{n}{krajni\PYZus{}zisk} \PY{o}{+}\PY{o}{=} \PY{n}{h}\PY{p}{[}\PY{n}{i}\PY{p}{]}
			\PY{n}{krajni\PYZus{}delka} \PY{o}{+}\PY{o}{=} \PY{l+m+mi}{1}
		\PY{k}{else}\PY{p}{:}                      \PY{c}{\PYZsh{} Případ 3}
			\PY{n}{krajni\PYZus{}zisk} \PY{p}{,}\PY{n}{krajni\PYZus{}i}\PY{p}{,} \PY{n}{krajni\PYZus{}delka} \PY{o}{=} \PY{n}{h}\PY{p}{[}\PY{n}{i}\PY{p}{]}\PY{p}{,} \PY{n}{i}\PY{p}{,} \PY{l+m+mi}{1}

		\PY{k}{if} \PY{n}{krajni\PYZus{}zisk} \PY{o}{>} \PY{n}{max\PYZus{}zisk}\PY{p}{:} \PY{c}{\PYZsh{} Případ 1 (obráceně)}
			\PY{n}{max\PYZus{}zisk}\PY{p}{,} \PY{n}{max\PYZus{}i}\PY{p}{,} \PY{n}{max\PYZus{}delka} \PY{o}{=} \PY{n}{krajni\PYZus{}zisk}\PY{p}{,} \PY{n}{krajni\PYZus{}i}\PY{p}{,} \PY{n}{krajni\PYZus{}delka}
	\PY{k}{return} \PY{n}{max\PYZus{}zisk}\PY{p}{,} \PY{n}{max\PYZus{}i}\PY{p}{,} \PY{n}{max\PYZus{}delka}
\end{Verbatim}


\section{Implementace}
Program {\tt zisk.py} je v jazyce python 2.7.2. Vstupní data jsou kladná 
a záporná celá čísla. Výstupem je hodnota zisku, index první asistentky 
(od nuly), která začíná vybranou posloupnost a délka této posloupnosti. Jsou 
implementována řešení oběma metodami.

\subsection{Příklad použití}
\begin{verbatim}
# vygenerovani vstupnich dat:  300 cisel od -500 do 499
> test.data
for i in $(seq 1 300); do echo $(($RANDOM % 1000 - 500)) >> test.data; done

cat test.data | ./zisk.py
\end{verbatim}

\end{document}
 \\

\item krajní zisk s novou hodnotou je větší než $zisk(N)$, tedy \\
	$zisk(N + 1) = krajni\_zisk(N) + h(N + 1)$ \\
\documentclass[11pt]{article}
\usepackage[utf8]{inputenc}
\usepackage[czech]{babel}
\usepackage{a4wide}
\usepackage{amsmath}


\include{pygments}


\title{Programovací strategie:  Maximalizujte svůj zisk}
\author{Tomáš Maršálek}
\date{6.\,října 2011}

%% -----------------------------------------


\begin{document}
\maketitle

\section{Zadání}
Se soutěžemi se v televizi roztrhl pytel. Je na čase, abyste si vydělali také 
vy. Otázky jsou pro vás naprosto triviální a přichází čas vybrat si odměnu. 
Přivedou před vás n asistentek. Asistentky se drží za ruce a každá má na krku 
pověšenou ceduli s celočíselnou sumou, kterou získáte (kladné číslo) nebo 
naopak ztratíte (záporné číslo), když si tuto asistentku vyberete. Můžete si 
vybrat jakýkoliv počet asistentek, ale jejich spojené ruce smíte přerušit max. 
na dvou místech, tedy posloupnost odměn, kterou si vyberete, musí být spojitá, 
ale může být libovolně dlouhá. Asistentky nezískáváte, pouze sumy výher, 
takže jiná než finanční kritéria není třeba brát v úvahu.

\section{Naivní řešení}
Jednoduše vyzkoušíme všechny souvislé podposloupnosti a vybereme tu s nejvyšší
sumou. Tedy pro všechny možné délky podposloupnosti a pro všechny jejich možné 
posuny zkoušíme, jestli je jejich suma lepší než dosavadní.
\begin{Verbatim}[commandchars=\\\{\}]
\PY{k}{def} \PY{n+nf}{naivniZisk}\PY{p}{(}\PY{n}{h}\PY{p}{)}\PY{p}{:}
	\PY{n}{n} \PY{o}{=} \PY{n+nb}{len}\PY{p}{(}\PY{n}{h}\PY{p}{)}
	\PY{n}{max\PYZus{}zisk}\PY{p}{,} \PY{n}{max\PYZus{}i}\PY{p}{,} \PY{n}{max\PYZus{}delka} \PY{o}{=} \PY{l+m+mi}{0}\PY{p}{,} \PY{l+m+mi}{0}\PY{p}{,} \PY{l+m+mi}{0}
	\PY{n}{pocatecni\PYZus{}zisk} \PY{o}{=} \PY{l+m+mi}{0}
	\PY{k}{for} \PY{n}{delka} \PY{o+ow}{in} \PY{n+nb}{range}\PY{p}{(}\PY{l+m+mi}{1}\PY{p}{,} \PY{n}{n} \PY{o}{+} \PY{l+m+mi}{1}\PY{p}{)}\PY{p}{:}
		\PY{n}{pocatecni\PYZus{}zisk} \PY{o}{+}\PY{o}{=} \PY{n}{h}\PY{p}{[}\PY{n}{delka} \PY{o}{-} \PY{l+m+mi}{1}\PY{p}{]}
		\PY{n}{zisk} \PY{o}{=} \PY{n}{pocatecni\PYZus{}zisk}

		\PY{k}{for} \PY{n}{i} \PY{o+ow}{in} \PY{n+nb}{range}\PY{p}{(}\PY{n}{n} \PY{o}{-} \PY{n}{delka}\PY{p}{)}\PY{p}{:}
			\PY{k}{if} \PY{n}{zisk} \PY{o}{>} \PY{n}{max\PYZus{}zisk}\PY{p}{:}
				\PY{n}{max\PYZus{}zisk}\PY{p}{,} \PY{n}{max\PYZus{}i}\PY{p}{,} \PY{n}{max\PYZus{}delka} \PY{o}{=} \PY{n}{zisk}\PY{p}{,} \PY{n}{i}\PY{p}{,} \PY{n}{delka}
			\PY{n}{zisk} \PY{o}{+}\PY{o}{=} \PY{n}{h}\PY{p}{[}\PY{n}{delka} \PY{o}{+} \PY{n}{i}\PY{p}{]} \PY{o}{-} \PY{n}{h}\PY{p}{[}\PY{n}{i}\PY{p}{]}

		\PY{k}{if} \PY{n}{zisk} \PY{o}{>} \PY{n}{max\PYZus{}zisk}\PY{p}{:}
			\PY{n}{max\PYZus{}zisk}\PY{p}{,} \PY{n}{max\PYZus{}i}\PY{p}{,} \PY{n}{max\PYZus{}delka} \PY{o}{=} \PY{n}{zisk}\PY{p}{,} \PY{n}{n} \PY{o}{-} \PY{n}{delka}\PY{p}{,} \PY{n}{delka}
	\PY{k}{return} \PY{n}{max\PYZus{}zisk}\PY{p}{,} \PY{n}{max\PYZus{}i}\PY{p}{,} \PY{n}{max\PYZus{}delka}
\end{Verbatim}


\subsection{Složitost}
Máme dvojitě vnořený cyklus, očekáváme kvadratickou složitost.
\begin{align*}
T(N) &= (\Theta(1) + (N - 1)) + (\Theta(1) + (N - 2)) + \ldots + (\Theta(1) + (N - N)) \\
&= \Theta(N) + N^2 - \frac{N}{2}(N + 1) \\
&= \Theta (N^2)
\end{align*}

\vspace{1cm}

\section{Dynamické programování}
Definujme funkci zisk, kterou chceme maximalizovat pro posloupnost délky N
$$zisk(N) = nejvy\check{s}\check{s}\acute{\i}\ suma\ v\ cel\acute{e}\  
posloupnosti \\ $$

\setlength{\unitlength}{.6mm}
\begin{picture}(150, 15)(-30, 0)

\put(-30, 3.5){\em N:}
\thinlines
\put(0, 0){\line(1, 0){150}}
\put(0, 10){\line(1, 0){150}}
\put(0, 0){\line(0, 1){10}}
\put(150, 0){\line(0, 1){10}}

\linethickness{.5mm}
% zisk(N)
\put(35, 3.5){\em zisk(N)}
\put(20, 0){\line(1, 0){50}}
\put(20, 10){\line(1, 0){50}}
\put(20, 0){\line(0, 1){10}}
\put(70, 0){\line(0, 1){10}}

\end{picture}
 \\

\vspace{1cm}

Abychom našli algoritmus založený na dynamickém programování, hledáme rekurzivní
vyjádření této funkce. Dodefinujeme ještě pomocné funkce

\begin{align*}
krajni\_zisk(N) &= nejvy\check{s}\check{s}\acute{\i}\ zisk\ z\ 
posloupnosti\ u\ prav\acute{e}ho\ kraje \\
h(N) &= hodnota\ N-t\acute{e}ho\ prvku
\end{align*}
\mbox{}

\begin{picture}(150, 15)(-30, 0)

%% figure 1 %%
\put(-30, 3.5){\em N:}

\thinlines
\put(0, 0){\line(1, 0){150}}
\put(0, 10){\line(1, 0){150}}
\put(0, 0){\line(0, 1){10}}
\put(150, 0){\line(0, 1){10}}

\linethickness{.5mm}
% zisk(N)
\put(35, 3.5){\em zisk(N)}
\put(20, 0){\line(1, 0){50}}
\put(20, 10){\line(1, 0){50}}
\put(20, 0){\line(0, 1){10}}
\put(70, 0){\line(0, 1){10}}

% krajni_max(N)
\put(96, 3.5){\em krajni\_zisk(N)}
\put(80, 0){\line(1, 0){70}}
\put(80, 10){\line(1, 0){70}}
\put(80, 0){\line(0, 1){10}}
\put(150, 0){\line(0, 1){10}}

\end{picture}
\mbox{}

\begin{picture}(150, 15)(-30, 0)

%% figure 2 %%
\put(-30, 3.5){\em N + 1:}
\thinlines
\put(0, 0){\line(1, 0){150}}
\put(0, 10){\line(1, 0){150}}
\put(0, 0){\line(0, 1){10}}
\put(150, 0){\line(0, 1){10}}

\linethickness{.5mm}
% zisk(N)
\put(35, 3.5){\em zisk(N)}
\put(20, 0){\line(1, 0){50}}
\put(20, 10){\line(1, 0){50}}
\put(20, 0){\line(0, 1){10}}
\put(70, 0){\line(0, 1){10}}

% krajni_max(N)
\put(96, 3.5){\em krajni\_zisk(N)}
\put(80, 0){\line(1, 0){70}}
\put(80, 10){\line(1, 0){70}}
\put(80, 0){\line(0, 1){10}}
\put(150, 0){\line(0, 1){10}}

% N + 1 hodnota
\put(155, 3.5){\em h(N + 1)}
\put(150, 0){\line(1, 0){35}}
\put(150, 10){\line(1, 0){35}}
\put(185, 0){\line(0, 1){10}}

\end{picture}


\mbox{}
\vspace{1cm}

\clearpage
Vyjádříme-li $zisk(N + 1)$ pomocí $zisk(N)$, mohou nastat tři případy.
\begin{enumerate}
\item krajní zisk nám vůbec nepomohl a $zisk(N + 1) = zisk(N)$ \\
\documentclass[11pt]{article}
\usepackage[utf8]{inputenc}
\usepackage[czech]{babel}
\usepackage{a4wide}
\usepackage{amsmath}


\include{pygments}


\title{Programovací strategie:  Maximalizujte svůj zisk}
\author{Tomáš Maršálek}
\date{6.\,října 2011}

%% -----------------------------------------


\begin{document}
\maketitle

\section{Zadání}
Se soutěžemi se v televizi roztrhl pytel. Je na čase, abyste si vydělali také 
vy. Otázky jsou pro vás naprosto triviální a přichází čas vybrat si odměnu. 
Přivedou před vás n asistentek. Asistentky se drží za ruce a každá má na krku 
pověšenou ceduli s celočíselnou sumou, kterou získáte (kladné číslo) nebo 
naopak ztratíte (záporné číslo), když si tuto asistentku vyberete. Můžete si 
vybrat jakýkoliv počet asistentek, ale jejich spojené ruce smíte přerušit max. 
na dvou místech, tedy posloupnost odměn, kterou si vyberete, musí být spojitá, 
ale může být libovolně dlouhá. Asistentky nezískáváte, pouze sumy výher, 
takže jiná než finanční kritéria není třeba brát v úvahu.

\section{Naivní řešení}
Jednoduše vyzkoušíme všechny souvislé podposloupnosti a vybereme tu s nejvyšší
sumou. Tedy pro všechny možné délky podposloupnosti a pro všechny jejich možné 
posuny zkoušíme, jestli je jejich suma lepší než dosavadní.
\begin{Verbatim}[commandchars=\\\{\}]
\PY{k}{def} \PY{n+nf}{naivniZisk}\PY{p}{(}\PY{n}{h}\PY{p}{)}\PY{p}{:}
	\PY{n}{n} \PY{o}{=} \PY{n+nb}{len}\PY{p}{(}\PY{n}{h}\PY{p}{)}
	\PY{n}{max\PYZus{}zisk}\PY{p}{,} \PY{n}{max\PYZus{}i}\PY{p}{,} \PY{n}{max\PYZus{}delka} \PY{o}{=} \PY{l+m+mi}{0}\PY{p}{,} \PY{l+m+mi}{0}\PY{p}{,} \PY{l+m+mi}{0}
	\PY{n}{pocatecni\PYZus{}zisk} \PY{o}{=} \PY{l+m+mi}{0}
	\PY{k}{for} \PY{n}{delka} \PY{o+ow}{in} \PY{n+nb}{range}\PY{p}{(}\PY{l+m+mi}{1}\PY{p}{,} \PY{n}{n} \PY{o}{+} \PY{l+m+mi}{1}\PY{p}{)}\PY{p}{:}
		\PY{n}{pocatecni\PYZus{}zisk} \PY{o}{+}\PY{o}{=} \PY{n}{h}\PY{p}{[}\PY{n}{delka} \PY{o}{-} \PY{l+m+mi}{1}\PY{p}{]}
		\PY{n}{zisk} \PY{o}{=} \PY{n}{pocatecni\PYZus{}zisk}

		\PY{k}{for} \PY{n}{i} \PY{o+ow}{in} \PY{n+nb}{range}\PY{p}{(}\PY{n}{n} \PY{o}{-} \PY{n}{delka}\PY{p}{)}\PY{p}{:}
			\PY{k}{if} \PY{n}{zisk} \PY{o}{>} \PY{n}{max\PYZus{}zisk}\PY{p}{:}
				\PY{n}{max\PYZus{}zisk}\PY{p}{,} \PY{n}{max\PYZus{}i}\PY{p}{,} \PY{n}{max\PYZus{}delka} \PY{o}{=} \PY{n}{zisk}\PY{p}{,} \PY{n}{i}\PY{p}{,} \PY{n}{delka}
			\PY{n}{zisk} \PY{o}{+}\PY{o}{=} \PY{n}{h}\PY{p}{[}\PY{n}{delka} \PY{o}{+} \PY{n}{i}\PY{p}{]} \PY{o}{-} \PY{n}{h}\PY{p}{[}\PY{n}{i}\PY{p}{]}

		\PY{k}{if} \PY{n}{zisk} \PY{o}{>} \PY{n}{max\PYZus{}zisk}\PY{p}{:}
			\PY{n}{max\PYZus{}zisk}\PY{p}{,} \PY{n}{max\PYZus{}i}\PY{p}{,} \PY{n}{max\PYZus{}delka} \PY{o}{=} \PY{n}{zisk}\PY{p}{,} \PY{n}{n} \PY{o}{-} \PY{n}{delka}\PY{p}{,} \PY{n}{delka}
	\PY{k}{return} \PY{n}{max\PYZus{}zisk}\PY{p}{,} \PY{n}{max\PYZus{}i}\PY{p}{,} \PY{n}{max\PYZus{}delka}
\end{Verbatim}


\subsection{Složitost}
Máme dvojitě vnořený cyklus, očekáváme kvadratickou složitost.
\begin{align*}
T(N) &= (\Theta(1) + (N - 1)) + (\Theta(1) + (N - 2)) + \ldots + (\Theta(1) + (N - N)) \\
&= \Theta(N) + N^2 - \frac{N}{2}(N + 1) \\
&= \Theta (N^2)
\end{align*}

\vspace{1cm}

\section{Dynamické programování}
Definujme funkci zisk, kterou chceme maximalizovat pro posloupnost délky N
$$zisk(N) = nejvy\check{s}\check{s}\acute{\i}\ suma\ v\ cel\acute{e}\  
posloupnosti \\ $$

\setlength{\unitlength}{.6mm}
\begin{picture}(150, 15)(-30, 0)

\put(-30, 3.5){\em N:}
\thinlines
\put(0, 0){\line(1, 0){150}}
\put(0, 10){\line(1, 0){150}}
\put(0, 0){\line(0, 1){10}}
\put(150, 0){\line(0, 1){10}}

\linethickness{.5mm}
% zisk(N)
\put(35, 3.5){\em zisk(N)}
\put(20, 0){\line(1, 0){50}}
\put(20, 10){\line(1, 0){50}}
\put(20, 0){\line(0, 1){10}}
\put(70, 0){\line(0, 1){10}}

\end{picture}
 \\

\vspace{1cm}

Abychom našli algoritmus založený na dynamickém programování, hledáme rekurzivní
vyjádření této funkce. Dodefinujeme ještě pomocné funkce

\begin{align*}
krajni\_zisk(N) &= nejvy\check{s}\check{s}\acute{\i}\ zisk\ z\ 
posloupnosti\ u\ prav\acute{e}ho\ kraje \\
h(N) &= hodnota\ N-t\acute{e}ho\ prvku
\end{align*}
\mbox{}

\begin{picture}(150, 15)(-30, 0)

%% figure 1 %%
\put(-30, 3.5){\em N:}

\thinlines
\put(0, 0){\line(1, 0){150}}
\put(0, 10){\line(1, 0){150}}
\put(0, 0){\line(0, 1){10}}
\put(150, 0){\line(0, 1){10}}

\linethickness{.5mm}
% zisk(N)
\put(35, 3.5){\em zisk(N)}
\put(20, 0){\line(1, 0){50}}
\put(20, 10){\line(1, 0){50}}
\put(20, 0){\line(0, 1){10}}
\put(70, 0){\line(0, 1){10}}

% krajni_max(N)
\put(96, 3.5){\em krajni\_zisk(N)}
\put(80, 0){\line(1, 0){70}}
\put(80, 10){\line(1, 0){70}}
\put(80, 0){\line(0, 1){10}}
\put(150, 0){\line(0, 1){10}}

\end{picture}
\mbox{}

\begin{picture}(150, 15)(-30, 0)

%% figure 2 %%
\put(-30, 3.5){\em N + 1:}
\thinlines
\put(0, 0){\line(1, 0){150}}
\put(0, 10){\line(1, 0){150}}
\put(0, 0){\line(0, 1){10}}
\put(150, 0){\line(0, 1){10}}

\linethickness{.5mm}
% zisk(N)
\put(35, 3.5){\em zisk(N)}
\put(20, 0){\line(1, 0){50}}
\put(20, 10){\line(1, 0){50}}
\put(20, 0){\line(0, 1){10}}
\put(70, 0){\line(0, 1){10}}

% krajni_max(N)
\put(96, 3.5){\em krajni\_zisk(N)}
\put(80, 0){\line(1, 0){70}}
\put(80, 10){\line(1, 0){70}}
\put(80, 0){\line(0, 1){10}}
\put(150, 0){\line(0, 1){10}}

% N + 1 hodnota
\put(155, 3.5){\em h(N + 1)}
\put(150, 0){\line(1, 0){35}}
\put(150, 10){\line(1, 0){35}}
\put(185, 0){\line(0, 1){10}}

\end{picture}


\mbox{}
\vspace{1cm}

\clearpage
Vyjádříme-li $zisk(N + 1)$ pomocí $zisk(N)$, mohou nastat tři případy.
\begin{enumerate}
\item krajní zisk nám vůbec nepomohl a $zisk(N + 1) = zisk(N)$ \\
\documentclass[11pt]{article}
\usepackage[utf8]{inputenc}
\usepackage[czech]{babel}
\usepackage{a4wide}
\usepackage{amsmath}


\include{pygments}


\title{Programovací strategie:  Maximalizujte svůj zisk}
\author{Tomáš Maršálek}
\date{6.\,října 2011}

%% -----------------------------------------


\begin{document}
\maketitle

\section{Zadání}
Se soutěžemi se v televizi roztrhl pytel. Je na čase, abyste si vydělali také 
vy. Otázky jsou pro vás naprosto triviální a přichází čas vybrat si odměnu. 
Přivedou před vás n asistentek. Asistentky se drží za ruce a každá má na krku 
pověšenou ceduli s celočíselnou sumou, kterou získáte (kladné číslo) nebo 
naopak ztratíte (záporné číslo), když si tuto asistentku vyberete. Můžete si 
vybrat jakýkoliv počet asistentek, ale jejich spojené ruce smíte přerušit max. 
na dvou místech, tedy posloupnost odměn, kterou si vyberete, musí být spojitá, 
ale může být libovolně dlouhá. Asistentky nezískáváte, pouze sumy výher, 
takže jiná než finanční kritéria není třeba brát v úvahu.

\section{Naivní řešení}
Jednoduše vyzkoušíme všechny souvislé podposloupnosti a vybereme tu s nejvyšší
sumou. Tedy pro všechny možné délky podposloupnosti a pro všechny jejich možné 
posuny zkoušíme, jestli je jejich suma lepší než dosavadní.
\input{naivniZisk.py}

\subsection{Složitost}
Máme dvojitě vnořený cyklus, očekáváme kvadratickou složitost.
\begin{align*}
T(N) &= (\Theta(1) + (N - 1)) + (\Theta(1) + (N - 2)) + \ldots + (\Theta(1) + (N - N)) \\
&= \Theta(N) + N^2 - \frac{N}{2}(N + 1) \\
&= \Theta (N^2)
\end{align*}

\vspace{1cm}

\section{Dynamické programování}
Definujme funkci zisk, kterou chceme maximalizovat pro posloupnost délky N
$$zisk(N) = nejvy\check{s}\check{s}\acute{\i}\ suma\ v\ cel\acute{e}\  
posloupnosti \\ $$

\setlength{\unitlength}{.6mm}
\input{zisk.fig1} \\

\vspace{1cm}

Abychom našli algoritmus založený na dynamickém programování, hledáme rekurzivní
vyjádření této funkce. Dodefinujeme ještě pomocné funkce

\begin{align*}
krajni\_zisk(N) &= nejvy\check{s}\check{s}\acute{\i}\ zisk\ z\ 
posloupnosti\ u\ prav\acute{e}ho\ kraje \\
h(N) &= hodnota\ N-t\acute{e}ho\ prvku
\end{align*}
\input{zisk.fig2}

\mbox{}
\vspace{1cm}

\clearpage
Vyjádříme-li $zisk(N + 1)$ pomocí $zisk(N)$, mohou nastat tři případy.
\begin{enumerate}
\item krajní zisk nám vůbec nepomohl a $zisk(N + 1) = zisk(N)$ \\
\input{zisk.fig3} \\

\item krajní zisk s novou hodnotou je větší než $zisk(N)$, tedy \\
	$zisk(N + 1) = krajni\_zisk(N) + h(N + 1)$ \\
\input{zisk.fig4} \\

\item $krajni\_zisk(N)$ je záporný, tedy je výhodnější vzít pouze 
	poslední hodnotu. \\
	$zisk(N + 1) = h(N + 1)$ \\
\input{zisk.fig5} \\

\end{enumerate} 

\vspace{1cm}

Rekurzivním vyjádřením pak dostáváme základ pro algoritmus. Rekurzi pouze 
obrátíme na dynamické programování.
\begin{align*}
zisk(N + 1) = \max (&zisk(N), \\
&krajni\_zisk(N) + h(N + 1), \\
&h(N + 1)) \\
\end{align*}

Stejně jako u předešlého algoritmu si zaznamenáváme hodnotu zisku, a polohu
podposloupnosti (počáteční index a délku). Náročnost je očividně $\Theta(N)$.
\input{dpZisk.py}

\section{Implementace}
Program {\tt zisk.py} je v jazyce python 2.7.2. Vstupní data jsou kladná 
a záporná celá čísla. Výstupem je hodnota zisku, index první asistentky 
(od nuly), která začíná vybranou posloupnost a délka této posloupnosti. Jsou 
implementována řešení oběma metodami.

\subsection{Příklad použití}
\begin{verbatim}
# vygenerovani vstupnich dat:  300 cisel od -500 do 499
> test.data
for i in $(seq 1 300); do echo $(($RANDOM % 1000 - 500)) >> test.data; done

cat test.data | ./zisk.py
\end{verbatim}

\end{document}
 \\

\item krajní zisk s novou hodnotou je větší než $zisk(N)$, tedy \\
	$zisk(N + 1) = krajni\_zisk(N) + h(N + 1)$ \\
\documentclass[11pt]{article}
\usepackage[utf8]{inputenc}
\usepackage[czech]{babel}
\usepackage{a4wide}
\usepackage{amsmath}


\include{pygments}


\title{Programovací strategie:  Maximalizujte svůj zisk}
\author{Tomáš Maršálek}
\date{6.\,října 2011}

%% -----------------------------------------


\begin{document}
\maketitle

\section{Zadání}
Se soutěžemi se v televizi roztrhl pytel. Je na čase, abyste si vydělali také 
vy. Otázky jsou pro vás naprosto triviální a přichází čas vybrat si odměnu. 
Přivedou před vás n asistentek. Asistentky se drží za ruce a každá má na krku 
pověšenou ceduli s celočíselnou sumou, kterou získáte (kladné číslo) nebo 
naopak ztratíte (záporné číslo), když si tuto asistentku vyberete. Můžete si 
vybrat jakýkoliv počet asistentek, ale jejich spojené ruce smíte přerušit max. 
na dvou místech, tedy posloupnost odměn, kterou si vyberete, musí být spojitá, 
ale může být libovolně dlouhá. Asistentky nezískáváte, pouze sumy výher, 
takže jiná než finanční kritéria není třeba brát v úvahu.

\section{Naivní řešení}
Jednoduše vyzkoušíme všechny souvislé podposloupnosti a vybereme tu s nejvyšší
sumou. Tedy pro všechny možné délky podposloupnosti a pro všechny jejich možné 
posuny zkoušíme, jestli je jejich suma lepší než dosavadní.
\input{naivniZisk.py}

\subsection{Složitost}
Máme dvojitě vnořený cyklus, očekáváme kvadratickou složitost.
\begin{align*}
T(N) &= (\Theta(1) + (N - 1)) + (\Theta(1) + (N - 2)) + \ldots + (\Theta(1) + (N - N)) \\
&= \Theta(N) + N^2 - \frac{N}{2}(N + 1) \\
&= \Theta (N^2)
\end{align*}

\vspace{1cm}

\section{Dynamické programování}
Definujme funkci zisk, kterou chceme maximalizovat pro posloupnost délky N
$$zisk(N) = nejvy\check{s}\check{s}\acute{\i}\ suma\ v\ cel\acute{e}\  
posloupnosti \\ $$

\setlength{\unitlength}{.6mm}
\input{zisk.fig1} \\

\vspace{1cm}

Abychom našli algoritmus založený na dynamickém programování, hledáme rekurzivní
vyjádření této funkce. Dodefinujeme ještě pomocné funkce

\begin{align*}
krajni\_zisk(N) &= nejvy\check{s}\check{s}\acute{\i}\ zisk\ z\ 
posloupnosti\ u\ prav\acute{e}ho\ kraje \\
h(N) &= hodnota\ N-t\acute{e}ho\ prvku
\end{align*}
\input{zisk.fig2}

\mbox{}
\vspace{1cm}

\clearpage
Vyjádříme-li $zisk(N + 1)$ pomocí $zisk(N)$, mohou nastat tři případy.
\begin{enumerate}
\item krajní zisk nám vůbec nepomohl a $zisk(N + 1) = zisk(N)$ \\
\input{zisk.fig3} \\

\item krajní zisk s novou hodnotou je větší než $zisk(N)$, tedy \\
	$zisk(N + 1) = krajni\_zisk(N) + h(N + 1)$ \\
\input{zisk.fig4} \\

\item $krajni\_zisk(N)$ je záporný, tedy je výhodnější vzít pouze 
	poslední hodnotu. \\
	$zisk(N + 1) = h(N + 1)$ \\
\input{zisk.fig5} \\

\end{enumerate} 

\vspace{1cm}

Rekurzivním vyjádřením pak dostáváme základ pro algoritmus. Rekurzi pouze 
obrátíme na dynamické programování.
\begin{align*}
zisk(N + 1) = \max (&zisk(N), \\
&krajni\_zisk(N) + h(N + 1), \\
&h(N + 1)) \\
\end{align*}

Stejně jako u předešlého algoritmu si zaznamenáváme hodnotu zisku, a polohu
podposloupnosti (počáteční index a délku). Náročnost je očividně $\Theta(N)$.
\input{dpZisk.py}

\section{Implementace}
Program {\tt zisk.py} je v jazyce python 2.7.2. Vstupní data jsou kladná 
a záporná celá čísla. Výstupem je hodnota zisku, index první asistentky 
(od nuly), která začíná vybranou posloupnost a délka této posloupnosti. Jsou 
implementována řešení oběma metodami.

\subsection{Příklad použití}
\begin{verbatim}
# vygenerovani vstupnich dat:  300 cisel od -500 do 499
> test.data
for i in $(seq 1 300); do echo $(($RANDOM % 1000 - 500)) >> test.data; done

cat test.data | ./zisk.py
\end{verbatim}

\end{document}
 \\

\item $krajni\_zisk(N)$ je záporný, tedy je výhodnější vzít pouze 
	poslední hodnotu. \\
	$zisk(N + 1) = h(N + 1)$ \\
\begin{picture}(150, 15)(-30, 0)

\put(-30, 3.5){\em N + 1:}
\thinlines
\put(0, 0){\line(1, 0){185}}
\put(0, 10){\line(1, 0){185}}
\put(0, 0){\line(0, 1){10}}
\put(185, 0){\line(0, 1){10}}

\linethickness{.5mm}

% N + 1 hodnota
\put(155, 3.5){\em h(N + 1)}
\put(150, 0){\line(1, 0){35}}
\put(150, 10){\line(1, 0){35}}
\put(185, 0){\line(0, 1){10}}
\put(150, 0){\line(0, 1){10}}

\end{picture}
 \\

\end{enumerate} 

\vspace{1cm}

Rekurzivním vyjádřením pak dostáváme základ pro algoritmus. Rekurzi pouze 
obrátíme na dynamické programování.
\begin{align*}
zisk(N + 1) = \max (&zisk(N), \\
&krajni\_zisk(N) + h(N + 1), \\
&h(N + 1)) \\
\end{align*}

Stejně jako u předešlého algoritmu si zaznamenáváme hodnotu zisku, a polohu
podposloupnosti (počáteční index a délku). Náročnost je očividně $\Theta(N)$.
\begin{Verbatim}[commandchars=\\\{\}]
\PY{k}{def} \PY{n+nf}{dpZisk}\PY{p}{(}\PY{n}{h}\PY{p}{)}\PY{p}{:}
	\PY{n}{n} \PY{o}{=} \PY{n+nb}{len}\PY{p}{(}\PY{n}{h}\PY{p}{)}
	\PY{n}{max\PYZus{}zisk}\PY{p}{,} \PY{n}{max\PYZus{}i}\PY{p}{,} \PY{n}{max\PYZus{}delka} \PY{o}{=} \PY{l+m+mi}{0}\PY{p}{,} \PY{l+m+mi}{0}\PY{p}{,} \PY{l+m+mi}{0}
	\PY{n}{krajni\PYZus{}zisk}\PY{p}{,} \PY{n}{krajni\PYZus{}i}\PY{p}{,} \PY{n}{krajni\PYZus{}delka} \PY{o}{=} \PY{l+m+mi}{0}\PY{p}{,} \PY{l+m+mi}{0}\PY{p}{,} \PY{l+m+mi}{0}
	\PY{k}{for} \PY{n}{i} \PY{o+ow}{in} \PY{n+nb}{range}\PY{p}{(}\PY{n}{n}\PY{p}{)}\PY{p}{:}
		\PY{k}{if} \PY{n}{krajni\PYZus{}zisk} \PY{o}{>} \PY{l+m+mi}{0}\PY{p}{:}        \PY{c}{\PYZsh{} Případ 2}
			\PY{n}{krajni\PYZus{}zisk} \PY{o}{+}\PY{o}{=} \PY{n}{h}\PY{p}{[}\PY{n}{i}\PY{p}{]}
			\PY{n}{krajni\PYZus{}delka} \PY{o}{+}\PY{o}{=} \PY{l+m+mi}{1}
		\PY{k}{else}\PY{p}{:}                      \PY{c}{\PYZsh{} Případ 3}
			\PY{n}{krajni\PYZus{}zisk} \PY{p}{,}\PY{n}{krajni\PYZus{}i}\PY{p}{,} \PY{n}{krajni\PYZus{}delka} \PY{o}{=} \PY{n}{h}\PY{p}{[}\PY{n}{i}\PY{p}{]}\PY{p}{,} \PY{n}{i}\PY{p}{,} \PY{l+m+mi}{1}

		\PY{k}{if} \PY{n}{krajni\PYZus{}zisk} \PY{o}{>} \PY{n}{max\PYZus{}zisk}\PY{p}{:} \PY{c}{\PYZsh{} Případ 1 (obráceně)}
			\PY{n}{max\PYZus{}zisk}\PY{p}{,} \PY{n}{max\PYZus{}i}\PY{p}{,} \PY{n}{max\PYZus{}delka} \PY{o}{=} \PY{n}{krajni\PYZus{}zisk}\PY{p}{,} \PY{n}{krajni\PYZus{}i}\PY{p}{,} \PY{n}{krajni\PYZus{}delka}
	\PY{k}{return} \PY{n}{max\PYZus{}zisk}\PY{p}{,} \PY{n}{max\PYZus{}i}\PY{p}{,} \PY{n}{max\PYZus{}delka}
\end{Verbatim}


\section{Implementace}
Program {\tt zisk.py} je v jazyce python 2.7.2. Vstupní data jsou kladná 
a záporná celá čísla. Výstupem je hodnota zisku, index první asistentky 
(od nuly), která začíná vybranou posloupnost a délka této posloupnosti. Jsou 
implementována řešení oběma metodami.

\subsection{Příklad použití}
\begin{verbatim}
# vygenerovani vstupnich dat:  300 cisel od -500 do 499
> test.data
for i in $(seq 1 300); do echo $(($RANDOM % 1000 - 500)) >> test.data; done

cat test.data | ./zisk.py
\end{verbatim}

\end{document}
 \\

\item krajní zisk s novou hodnotou je větší než $zisk(N)$, tedy \\
	$zisk(N + 1) = krajni\_zisk(N) + h(N + 1)$ \\
\documentclass[11pt]{article}
\usepackage[utf8]{inputenc}
\usepackage[czech]{babel}
\usepackage{a4wide}
\usepackage{amsmath}


\include{pygments}


\title{Programovací strategie:  Maximalizujte svůj zisk}
\author{Tomáš Maršálek}
\date{6.\,října 2011}

%% -----------------------------------------


\begin{document}
\maketitle

\section{Zadání}
Se soutěžemi se v televizi roztrhl pytel. Je na čase, abyste si vydělali také 
vy. Otázky jsou pro vás naprosto triviální a přichází čas vybrat si odměnu. 
Přivedou před vás n asistentek. Asistentky se drží za ruce a každá má na krku 
pověšenou ceduli s celočíselnou sumou, kterou získáte (kladné číslo) nebo 
naopak ztratíte (záporné číslo), když si tuto asistentku vyberete. Můžete si 
vybrat jakýkoliv počet asistentek, ale jejich spojené ruce smíte přerušit max. 
na dvou místech, tedy posloupnost odměn, kterou si vyberete, musí být spojitá, 
ale může být libovolně dlouhá. Asistentky nezískáváte, pouze sumy výher, 
takže jiná než finanční kritéria není třeba brát v úvahu.

\section{Naivní řešení}
Jednoduše vyzkoušíme všechny souvislé podposloupnosti a vybereme tu s nejvyšší
sumou. Tedy pro všechny možné délky podposloupnosti a pro všechny jejich možné 
posuny zkoušíme, jestli je jejich suma lepší než dosavadní.
\begin{Verbatim}[commandchars=\\\{\}]
\PY{k}{def} \PY{n+nf}{naivniZisk}\PY{p}{(}\PY{n}{h}\PY{p}{)}\PY{p}{:}
	\PY{n}{n} \PY{o}{=} \PY{n+nb}{len}\PY{p}{(}\PY{n}{h}\PY{p}{)}
	\PY{n}{max\PYZus{}zisk}\PY{p}{,} \PY{n}{max\PYZus{}i}\PY{p}{,} \PY{n}{max\PYZus{}delka} \PY{o}{=} \PY{l+m+mi}{0}\PY{p}{,} \PY{l+m+mi}{0}\PY{p}{,} \PY{l+m+mi}{0}
	\PY{n}{pocatecni\PYZus{}zisk} \PY{o}{=} \PY{l+m+mi}{0}
	\PY{k}{for} \PY{n}{delka} \PY{o+ow}{in} \PY{n+nb}{range}\PY{p}{(}\PY{l+m+mi}{1}\PY{p}{,} \PY{n}{n} \PY{o}{+} \PY{l+m+mi}{1}\PY{p}{)}\PY{p}{:}
		\PY{n}{pocatecni\PYZus{}zisk} \PY{o}{+}\PY{o}{=} \PY{n}{h}\PY{p}{[}\PY{n}{delka} \PY{o}{-} \PY{l+m+mi}{1}\PY{p}{]}
		\PY{n}{zisk} \PY{o}{=} \PY{n}{pocatecni\PYZus{}zisk}

		\PY{k}{for} \PY{n}{i} \PY{o+ow}{in} \PY{n+nb}{range}\PY{p}{(}\PY{n}{n} \PY{o}{-} \PY{n}{delka}\PY{p}{)}\PY{p}{:}
			\PY{k}{if} \PY{n}{zisk} \PY{o}{>} \PY{n}{max\PYZus{}zisk}\PY{p}{:}
				\PY{n}{max\PYZus{}zisk}\PY{p}{,} \PY{n}{max\PYZus{}i}\PY{p}{,} \PY{n}{max\PYZus{}delka} \PY{o}{=} \PY{n}{zisk}\PY{p}{,} \PY{n}{i}\PY{p}{,} \PY{n}{delka}
			\PY{n}{zisk} \PY{o}{+}\PY{o}{=} \PY{n}{h}\PY{p}{[}\PY{n}{delka} \PY{o}{+} \PY{n}{i}\PY{p}{]} \PY{o}{-} \PY{n}{h}\PY{p}{[}\PY{n}{i}\PY{p}{]}

		\PY{k}{if} \PY{n}{zisk} \PY{o}{>} \PY{n}{max\PYZus{}zisk}\PY{p}{:}
			\PY{n}{max\PYZus{}zisk}\PY{p}{,} \PY{n}{max\PYZus{}i}\PY{p}{,} \PY{n}{max\PYZus{}delka} \PY{o}{=} \PY{n}{zisk}\PY{p}{,} \PY{n}{n} \PY{o}{-} \PY{n}{delka}\PY{p}{,} \PY{n}{delka}
	\PY{k}{return} \PY{n}{max\PYZus{}zisk}\PY{p}{,} \PY{n}{max\PYZus{}i}\PY{p}{,} \PY{n}{max\PYZus{}delka}
\end{Verbatim}


\subsection{Složitost}
Máme dvojitě vnořený cyklus, očekáváme kvadratickou složitost.
\begin{align*}
T(N) &= (\Theta(1) + (N - 1)) + (\Theta(1) + (N - 2)) + \ldots + (\Theta(1) + (N - N)) \\
&= \Theta(N) + N^2 - \frac{N}{2}(N + 1) \\
&= \Theta (N^2)
\end{align*}

\vspace{1cm}

\section{Dynamické programování}
Definujme funkci zisk, kterou chceme maximalizovat pro posloupnost délky N
$$zisk(N) = nejvy\check{s}\check{s}\acute{\i}\ suma\ v\ cel\acute{e}\  
posloupnosti \\ $$

\setlength{\unitlength}{.6mm}
\begin{picture}(150, 15)(-30, 0)

\put(-30, 3.5){\em N:}
\thinlines
\put(0, 0){\line(1, 0){150}}
\put(0, 10){\line(1, 0){150}}
\put(0, 0){\line(0, 1){10}}
\put(150, 0){\line(0, 1){10}}

\linethickness{.5mm}
% zisk(N)
\put(35, 3.5){\em zisk(N)}
\put(20, 0){\line(1, 0){50}}
\put(20, 10){\line(1, 0){50}}
\put(20, 0){\line(0, 1){10}}
\put(70, 0){\line(0, 1){10}}

\end{picture}
 \\

\vspace{1cm}

Abychom našli algoritmus založený na dynamickém programování, hledáme rekurzivní
vyjádření této funkce. Dodefinujeme ještě pomocné funkce

\begin{align*}
krajni\_zisk(N) &= nejvy\check{s}\check{s}\acute{\i}\ zisk\ z\ 
posloupnosti\ u\ prav\acute{e}ho\ kraje \\
h(N) &= hodnota\ N-t\acute{e}ho\ prvku
\end{align*}
\mbox{}

\begin{picture}(150, 15)(-30, 0)

%% figure 1 %%
\put(-30, 3.5){\em N:}

\thinlines
\put(0, 0){\line(1, 0){150}}
\put(0, 10){\line(1, 0){150}}
\put(0, 0){\line(0, 1){10}}
\put(150, 0){\line(0, 1){10}}

\linethickness{.5mm}
% zisk(N)
\put(35, 3.5){\em zisk(N)}
\put(20, 0){\line(1, 0){50}}
\put(20, 10){\line(1, 0){50}}
\put(20, 0){\line(0, 1){10}}
\put(70, 0){\line(0, 1){10}}

% krajni_max(N)
\put(96, 3.5){\em krajni\_zisk(N)}
\put(80, 0){\line(1, 0){70}}
\put(80, 10){\line(1, 0){70}}
\put(80, 0){\line(0, 1){10}}
\put(150, 0){\line(0, 1){10}}

\end{picture}
\mbox{}

\begin{picture}(150, 15)(-30, 0)

%% figure 2 %%
\put(-30, 3.5){\em N + 1:}
\thinlines
\put(0, 0){\line(1, 0){150}}
\put(0, 10){\line(1, 0){150}}
\put(0, 0){\line(0, 1){10}}
\put(150, 0){\line(0, 1){10}}

\linethickness{.5mm}
% zisk(N)
\put(35, 3.5){\em zisk(N)}
\put(20, 0){\line(1, 0){50}}
\put(20, 10){\line(1, 0){50}}
\put(20, 0){\line(0, 1){10}}
\put(70, 0){\line(0, 1){10}}

% krajni_max(N)
\put(96, 3.5){\em krajni\_zisk(N)}
\put(80, 0){\line(1, 0){70}}
\put(80, 10){\line(1, 0){70}}
\put(80, 0){\line(0, 1){10}}
\put(150, 0){\line(0, 1){10}}

% N + 1 hodnota
\put(155, 3.5){\em h(N + 1)}
\put(150, 0){\line(1, 0){35}}
\put(150, 10){\line(1, 0){35}}
\put(185, 0){\line(0, 1){10}}

\end{picture}


\mbox{}
\vspace{1cm}

\clearpage
Vyjádříme-li $zisk(N + 1)$ pomocí $zisk(N)$, mohou nastat tři případy.
\begin{enumerate}
\item krajní zisk nám vůbec nepomohl a $zisk(N + 1) = zisk(N)$ \\
\documentclass[11pt]{article}
\usepackage[utf8]{inputenc}
\usepackage[czech]{babel}
\usepackage{a4wide}
\usepackage{amsmath}


\include{pygments}


\title{Programovací strategie:  Maximalizujte svůj zisk}
\author{Tomáš Maršálek}
\date{6.\,října 2011}

%% -----------------------------------------


\begin{document}
\maketitle

\section{Zadání}
Se soutěžemi se v televizi roztrhl pytel. Je na čase, abyste si vydělali také 
vy. Otázky jsou pro vás naprosto triviální a přichází čas vybrat si odměnu. 
Přivedou před vás n asistentek. Asistentky se drží za ruce a každá má na krku 
pověšenou ceduli s celočíselnou sumou, kterou získáte (kladné číslo) nebo 
naopak ztratíte (záporné číslo), když si tuto asistentku vyberete. Můžete si 
vybrat jakýkoliv počet asistentek, ale jejich spojené ruce smíte přerušit max. 
na dvou místech, tedy posloupnost odměn, kterou si vyberete, musí být spojitá, 
ale může být libovolně dlouhá. Asistentky nezískáváte, pouze sumy výher, 
takže jiná než finanční kritéria není třeba brát v úvahu.

\section{Naivní řešení}
Jednoduše vyzkoušíme všechny souvislé podposloupnosti a vybereme tu s nejvyšší
sumou. Tedy pro všechny možné délky podposloupnosti a pro všechny jejich možné 
posuny zkoušíme, jestli je jejich suma lepší než dosavadní.
\input{naivniZisk.py}

\subsection{Složitost}
Máme dvojitě vnořený cyklus, očekáváme kvadratickou složitost.
\begin{align*}
T(N) &= (\Theta(1) + (N - 1)) + (\Theta(1) + (N - 2)) + \ldots + (\Theta(1) + (N - N)) \\
&= \Theta(N) + N^2 - \frac{N}{2}(N + 1) \\
&= \Theta (N^2)
\end{align*}

\vspace{1cm}

\section{Dynamické programování}
Definujme funkci zisk, kterou chceme maximalizovat pro posloupnost délky N
$$zisk(N) = nejvy\check{s}\check{s}\acute{\i}\ suma\ v\ cel\acute{e}\  
posloupnosti \\ $$

\setlength{\unitlength}{.6mm}
\input{zisk.fig1} \\

\vspace{1cm}

Abychom našli algoritmus založený na dynamickém programování, hledáme rekurzivní
vyjádření této funkce. Dodefinujeme ještě pomocné funkce

\begin{align*}
krajni\_zisk(N) &= nejvy\check{s}\check{s}\acute{\i}\ zisk\ z\ 
posloupnosti\ u\ prav\acute{e}ho\ kraje \\
h(N) &= hodnota\ N-t\acute{e}ho\ prvku
\end{align*}
\input{zisk.fig2}

\mbox{}
\vspace{1cm}

\clearpage
Vyjádříme-li $zisk(N + 1)$ pomocí $zisk(N)$, mohou nastat tři případy.
\begin{enumerate}
\item krajní zisk nám vůbec nepomohl a $zisk(N + 1) = zisk(N)$ \\
\input{zisk.fig3} \\

\item krajní zisk s novou hodnotou je větší než $zisk(N)$, tedy \\
	$zisk(N + 1) = krajni\_zisk(N) + h(N + 1)$ \\
\input{zisk.fig4} \\

\item $krajni\_zisk(N)$ je záporný, tedy je výhodnější vzít pouze 
	poslední hodnotu. \\
	$zisk(N + 1) = h(N + 1)$ \\
\input{zisk.fig5} \\

\end{enumerate} 

\vspace{1cm}

Rekurzivním vyjádřením pak dostáváme základ pro algoritmus. Rekurzi pouze 
obrátíme na dynamické programování.
\begin{align*}
zisk(N + 1) = \max (&zisk(N), \\
&krajni\_zisk(N) + h(N + 1), \\
&h(N + 1)) \\
\end{align*}

Stejně jako u předešlého algoritmu si zaznamenáváme hodnotu zisku, a polohu
podposloupnosti (počáteční index a délku). Náročnost je očividně $\Theta(N)$.
\input{dpZisk.py}

\section{Implementace}
Program {\tt zisk.py} je v jazyce python 2.7.2. Vstupní data jsou kladná 
a záporná celá čísla. Výstupem je hodnota zisku, index první asistentky 
(od nuly), která začíná vybranou posloupnost a délka této posloupnosti. Jsou 
implementována řešení oběma metodami.

\subsection{Příklad použití}
\begin{verbatim}
# vygenerovani vstupnich dat:  300 cisel od -500 do 499
> test.data
for i in $(seq 1 300); do echo $(($RANDOM % 1000 - 500)) >> test.data; done

cat test.data | ./zisk.py
\end{verbatim}

\end{document}
 \\

\item krajní zisk s novou hodnotou je větší než $zisk(N)$, tedy \\
	$zisk(N + 1) = krajni\_zisk(N) + h(N + 1)$ \\
\documentclass[11pt]{article}
\usepackage[utf8]{inputenc}
\usepackage[czech]{babel}
\usepackage{a4wide}
\usepackage{amsmath}


\include{pygments}


\title{Programovací strategie:  Maximalizujte svůj zisk}
\author{Tomáš Maršálek}
\date{6.\,října 2011}

%% -----------------------------------------


\begin{document}
\maketitle

\section{Zadání}
Se soutěžemi se v televizi roztrhl pytel. Je na čase, abyste si vydělali také 
vy. Otázky jsou pro vás naprosto triviální a přichází čas vybrat si odměnu. 
Přivedou před vás n asistentek. Asistentky se drží za ruce a každá má na krku 
pověšenou ceduli s celočíselnou sumou, kterou získáte (kladné číslo) nebo 
naopak ztratíte (záporné číslo), když si tuto asistentku vyberete. Můžete si 
vybrat jakýkoliv počet asistentek, ale jejich spojené ruce smíte přerušit max. 
na dvou místech, tedy posloupnost odměn, kterou si vyberete, musí být spojitá, 
ale může být libovolně dlouhá. Asistentky nezískáváte, pouze sumy výher, 
takže jiná než finanční kritéria není třeba brát v úvahu.

\section{Naivní řešení}
Jednoduše vyzkoušíme všechny souvislé podposloupnosti a vybereme tu s nejvyšší
sumou. Tedy pro všechny možné délky podposloupnosti a pro všechny jejich možné 
posuny zkoušíme, jestli je jejich suma lepší než dosavadní.
\input{naivniZisk.py}

\subsection{Složitost}
Máme dvojitě vnořený cyklus, očekáváme kvadratickou složitost.
\begin{align*}
T(N) &= (\Theta(1) + (N - 1)) + (\Theta(1) + (N - 2)) + \ldots + (\Theta(1) + (N - N)) \\
&= \Theta(N) + N^2 - \frac{N}{2}(N + 1) \\
&= \Theta (N^2)
\end{align*}

\vspace{1cm}

\section{Dynamické programování}
Definujme funkci zisk, kterou chceme maximalizovat pro posloupnost délky N
$$zisk(N) = nejvy\check{s}\check{s}\acute{\i}\ suma\ v\ cel\acute{e}\  
posloupnosti \\ $$

\setlength{\unitlength}{.6mm}
\input{zisk.fig1} \\

\vspace{1cm}

Abychom našli algoritmus založený na dynamickém programování, hledáme rekurzivní
vyjádření této funkce. Dodefinujeme ještě pomocné funkce

\begin{align*}
krajni\_zisk(N) &= nejvy\check{s}\check{s}\acute{\i}\ zisk\ z\ 
posloupnosti\ u\ prav\acute{e}ho\ kraje \\
h(N) &= hodnota\ N-t\acute{e}ho\ prvku
\end{align*}
\input{zisk.fig2}

\mbox{}
\vspace{1cm}

\clearpage
Vyjádříme-li $zisk(N + 1)$ pomocí $zisk(N)$, mohou nastat tři případy.
\begin{enumerate}
\item krajní zisk nám vůbec nepomohl a $zisk(N + 1) = zisk(N)$ \\
\input{zisk.fig3} \\

\item krajní zisk s novou hodnotou je větší než $zisk(N)$, tedy \\
	$zisk(N + 1) = krajni\_zisk(N) + h(N + 1)$ \\
\input{zisk.fig4} \\

\item $krajni\_zisk(N)$ je záporný, tedy je výhodnější vzít pouze 
	poslední hodnotu. \\
	$zisk(N + 1) = h(N + 1)$ \\
\input{zisk.fig5} \\

\end{enumerate} 

\vspace{1cm}

Rekurzivním vyjádřením pak dostáváme základ pro algoritmus. Rekurzi pouze 
obrátíme na dynamické programování.
\begin{align*}
zisk(N + 1) = \max (&zisk(N), \\
&krajni\_zisk(N) + h(N + 1), \\
&h(N + 1)) \\
\end{align*}

Stejně jako u předešlého algoritmu si zaznamenáváme hodnotu zisku, a polohu
podposloupnosti (počáteční index a délku). Náročnost je očividně $\Theta(N)$.
\input{dpZisk.py}

\section{Implementace}
Program {\tt zisk.py} je v jazyce python 2.7.2. Vstupní data jsou kladná 
a záporná celá čísla. Výstupem je hodnota zisku, index první asistentky 
(od nuly), která začíná vybranou posloupnost a délka této posloupnosti. Jsou 
implementována řešení oběma metodami.

\subsection{Příklad použití}
\begin{verbatim}
# vygenerovani vstupnich dat:  300 cisel od -500 do 499
> test.data
for i in $(seq 1 300); do echo $(($RANDOM % 1000 - 500)) >> test.data; done

cat test.data | ./zisk.py
\end{verbatim}

\end{document}
 \\

\item $krajni\_zisk(N)$ je záporný, tedy je výhodnější vzít pouze 
	poslední hodnotu. \\
	$zisk(N + 1) = h(N + 1)$ \\
\begin{picture}(150, 15)(-30, 0)

\put(-30, 3.5){\em N + 1:}
\thinlines
\put(0, 0){\line(1, 0){185}}
\put(0, 10){\line(1, 0){185}}
\put(0, 0){\line(0, 1){10}}
\put(185, 0){\line(0, 1){10}}

\linethickness{.5mm}

% N + 1 hodnota
\put(155, 3.5){\em h(N + 1)}
\put(150, 0){\line(1, 0){35}}
\put(150, 10){\line(1, 0){35}}
\put(185, 0){\line(0, 1){10}}
\put(150, 0){\line(0, 1){10}}

\end{picture}
 \\

\end{enumerate} 

\vspace{1cm}

Rekurzivním vyjádřením pak dostáváme základ pro algoritmus. Rekurzi pouze 
obrátíme na dynamické programování.
\begin{align*}
zisk(N + 1) = \max (&zisk(N), \\
&krajni\_zisk(N) + h(N + 1), \\
&h(N + 1)) \\
\end{align*}

Stejně jako u předešlého algoritmu si zaznamenáváme hodnotu zisku, a polohu
podposloupnosti (počáteční index a délku). Náročnost je očividně $\Theta(N)$.
\begin{Verbatim}[commandchars=\\\{\}]
\PY{k}{def} \PY{n+nf}{dpZisk}\PY{p}{(}\PY{n}{h}\PY{p}{)}\PY{p}{:}
	\PY{n}{n} \PY{o}{=} \PY{n+nb}{len}\PY{p}{(}\PY{n}{h}\PY{p}{)}
	\PY{n}{max\PYZus{}zisk}\PY{p}{,} \PY{n}{max\PYZus{}i}\PY{p}{,} \PY{n}{max\PYZus{}delka} \PY{o}{=} \PY{l+m+mi}{0}\PY{p}{,} \PY{l+m+mi}{0}\PY{p}{,} \PY{l+m+mi}{0}
	\PY{n}{krajni\PYZus{}zisk}\PY{p}{,} \PY{n}{krajni\PYZus{}i}\PY{p}{,} \PY{n}{krajni\PYZus{}delka} \PY{o}{=} \PY{l+m+mi}{0}\PY{p}{,} \PY{l+m+mi}{0}\PY{p}{,} \PY{l+m+mi}{0}
	\PY{k}{for} \PY{n}{i} \PY{o+ow}{in} \PY{n+nb}{range}\PY{p}{(}\PY{n}{n}\PY{p}{)}\PY{p}{:}
		\PY{k}{if} \PY{n}{krajni\PYZus{}zisk} \PY{o}{>} \PY{l+m+mi}{0}\PY{p}{:}        \PY{c}{\PYZsh{} Případ 2}
			\PY{n}{krajni\PYZus{}zisk} \PY{o}{+}\PY{o}{=} \PY{n}{h}\PY{p}{[}\PY{n}{i}\PY{p}{]}
			\PY{n}{krajni\PYZus{}delka} \PY{o}{+}\PY{o}{=} \PY{l+m+mi}{1}
		\PY{k}{else}\PY{p}{:}                      \PY{c}{\PYZsh{} Případ 3}
			\PY{n}{krajni\PYZus{}zisk} \PY{p}{,}\PY{n}{krajni\PYZus{}i}\PY{p}{,} \PY{n}{krajni\PYZus{}delka} \PY{o}{=} \PY{n}{h}\PY{p}{[}\PY{n}{i}\PY{p}{]}\PY{p}{,} \PY{n}{i}\PY{p}{,} \PY{l+m+mi}{1}

		\PY{k}{if} \PY{n}{krajni\PYZus{}zisk} \PY{o}{>} \PY{n}{max\PYZus{}zisk}\PY{p}{:} \PY{c}{\PYZsh{} Případ 1 (obráceně)}
			\PY{n}{max\PYZus{}zisk}\PY{p}{,} \PY{n}{max\PYZus{}i}\PY{p}{,} \PY{n}{max\PYZus{}delka} \PY{o}{=} \PY{n}{krajni\PYZus{}zisk}\PY{p}{,} \PY{n}{krajni\PYZus{}i}\PY{p}{,} \PY{n}{krajni\PYZus{}delka}
	\PY{k}{return} \PY{n}{max\PYZus{}zisk}\PY{p}{,} \PY{n}{max\PYZus{}i}\PY{p}{,} \PY{n}{max\PYZus{}delka}
\end{Verbatim}


\section{Implementace}
Program {\tt zisk.py} je v jazyce python 2.7.2. Vstupní data jsou kladná 
a záporná celá čísla. Výstupem je hodnota zisku, index první asistentky 
(od nuly), která začíná vybranou posloupnost a délka této posloupnosti. Jsou 
implementována řešení oběma metodami.

\subsection{Příklad použití}
\begin{verbatim}
# vygenerovani vstupnich dat:  300 cisel od -500 do 499
> test.data
for i in $(seq 1 300); do echo $(($RANDOM % 1000 - 500)) >> test.data; done

cat test.data | ./zisk.py
\end{verbatim}

\end{document}
 \\

\item $krajni\_zisk(N)$ je záporný, tedy je výhodnější vzít pouze 
	poslední hodnotu. \\
	$zisk(N + 1) = h(N + 1)$ \\
\begin{picture}(150, 15)(-30, 0)

\put(-30, 3.5){\em N + 1:}
\thinlines
\put(0, 0){\line(1, 0){185}}
\put(0, 10){\line(1, 0){185}}
\put(0, 0){\line(0, 1){10}}
\put(185, 0){\line(0, 1){10}}

\linethickness{.5mm}

% N + 1 hodnota
\put(155, 3.5){\em h(N + 1)}
\put(150, 0){\line(1, 0){35}}
\put(150, 10){\line(1, 0){35}}
\put(185, 0){\line(0, 1){10}}
\put(150, 0){\line(0, 1){10}}

\end{picture}
 \\

\end{enumerate} 

\vspace{1cm}

Rekurzivním vyjádřením pak dostáváme základ pro algoritmus. Rekurzi pouze 
obrátíme na dynamické programování.
\begin{align*}
zisk(N + 1) = \max (&zisk(N), \\
&krajni\_zisk(N) + h(N + 1), \\
&h(N + 1)) \\
\end{align*}

Stejně jako u předešlého algoritmu si zaznamenáváme hodnotu zisku, a polohu
podposloupnosti (počáteční index a délku). Náročnost je očividně $\Theta(N)$.
\begin{Verbatim}[commandchars=\\\{\}]
\PY{k}{def} \PY{n+nf}{dpZisk}\PY{p}{(}\PY{n}{h}\PY{p}{)}\PY{p}{:}
	\PY{n}{n} \PY{o}{=} \PY{n+nb}{len}\PY{p}{(}\PY{n}{h}\PY{p}{)}
	\PY{n}{max\PYZus{}zisk}\PY{p}{,} \PY{n}{max\PYZus{}i}\PY{p}{,} \PY{n}{max\PYZus{}delka} \PY{o}{=} \PY{l+m+mi}{0}\PY{p}{,} \PY{l+m+mi}{0}\PY{p}{,} \PY{l+m+mi}{0}
	\PY{n}{krajni\PYZus{}zisk}\PY{p}{,} \PY{n}{krajni\PYZus{}i}\PY{p}{,} \PY{n}{krajni\PYZus{}delka} \PY{o}{=} \PY{l+m+mi}{0}\PY{p}{,} \PY{l+m+mi}{0}\PY{p}{,} \PY{l+m+mi}{0}
	\PY{k}{for} \PY{n}{i} \PY{o+ow}{in} \PY{n+nb}{range}\PY{p}{(}\PY{n}{n}\PY{p}{)}\PY{p}{:}
		\PY{k}{if} \PY{n}{krajni\PYZus{}zisk} \PY{o}{>} \PY{l+m+mi}{0}\PY{p}{:}        \PY{c}{\PYZsh{} Případ 2}
			\PY{n}{krajni\PYZus{}zisk} \PY{o}{+}\PY{o}{=} \PY{n}{h}\PY{p}{[}\PY{n}{i}\PY{p}{]}
			\PY{n}{krajni\PYZus{}delka} \PY{o}{+}\PY{o}{=} \PY{l+m+mi}{1}
		\PY{k}{else}\PY{p}{:}                      \PY{c}{\PYZsh{} Případ 3}
			\PY{n}{krajni\PYZus{}zisk} \PY{p}{,}\PY{n}{krajni\PYZus{}i}\PY{p}{,} \PY{n}{krajni\PYZus{}delka} \PY{o}{=} \PY{n}{h}\PY{p}{[}\PY{n}{i}\PY{p}{]}\PY{p}{,} \PY{n}{i}\PY{p}{,} \PY{l+m+mi}{1}

		\PY{k}{if} \PY{n}{krajni\PYZus{}zisk} \PY{o}{>} \PY{n}{max\PYZus{}zisk}\PY{p}{:} \PY{c}{\PYZsh{} Případ 1 (obráceně)}
			\PY{n}{max\PYZus{}zisk}\PY{p}{,} \PY{n}{max\PYZus{}i}\PY{p}{,} \PY{n}{max\PYZus{}delka} \PY{o}{=} \PY{n}{krajni\PYZus{}zisk}\PY{p}{,} \PY{n}{krajni\PYZus{}i}\PY{p}{,} \PY{n}{krajni\PYZus{}delka}
	\PY{k}{return} \PY{n}{max\PYZus{}zisk}\PY{p}{,} \PY{n}{max\PYZus{}i}\PY{p}{,} \PY{n}{max\PYZus{}delka}
\end{Verbatim}


\section{Implementace}
Program {\tt zisk.py} je v jazyce python 2.7.2. Vstupní data jsou kladná 
a záporná celá čísla. Výstupem je hodnota zisku, index první asistentky 
(od nuly), která začíná vybranou posloupnost a délka této posloupnosti. Jsou 
implementována řešení oběma metodami.

\subsection{Příklad použití}
\begin{verbatim}
# vygenerovani vstupnich dat:  300 cisel od -500 do 499
> test.data
for i in $(seq 1 300); do echo $(($RANDOM % 1000 - 500)) >> test.data; done

cat test.data | ./zisk.py
\end{verbatim}

\end{document}
 \\

\item $krajni\_zisk(N)$ je záporný, tedy je výhodnější vzít pouze 
	poslední hodnotu. \\
	$zisk(N + 1) = h(N + 1)$ \\
\begin{picture}(150, 15)(-30, 0)

\put(-30, 3.5){\em N + 1:}
\thinlines
\put(0, 0){\line(1, 0){185}}
\put(0, 10){\line(1, 0){185}}
\put(0, 0){\line(0, 1){10}}
\put(185, 0){\line(0, 1){10}}

\linethickness{.5mm}

% N + 1 hodnota
\put(155, 3.5){\em h(N + 1)}
\put(150, 0){\line(1, 0){35}}
\put(150, 10){\line(1, 0){35}}
\put(185, 0){\line(0, 1){10}}
\put(150, 0){\line(0, 1){10}}

\end{picture}
 \\

\end{enumerate} 

\vspace{1cm}

Rekurzivním vyjádřením pak dostáváme základ pro algoritmus. Rekurzi pouze 
obrátíme na dynamické programování.
\begin{align*}
zisk(N + 1) = \max (&zisk(N), \\
&krajni\_zisk(N) + h(N + 1), \\
&h(N + 1)) \\
\end{align*}

Stejně jako u předešlého algoritmu si zaznamenáváme hodnotu zisku, a polohu
podposloupnosti (počáteční index a délku). Náročnost je očividně $\Theta(N)$.
\begin{Verbatim}[commandchars=\\\{\}]
\PY{k}{def} \PY{n+nf}{dpZisk}\PY{p}{(}\PY{n}{h}\PY{p}{)}\PY{p}{:}
	\PY{n}{n} \PY{o}{=} \PY{n+nb}{len}\PY{p}{(}\PY{n}{h}\PY{p}{)}
	\PY{n}{max\PYZus{}zisk}\PY{p}{,} \PY{n}{max\PYZus{}i}\PY{p}{,} \PY{n}{max\PYZus{}delka} \PY{o}{=} \PY{l+m+mi}{0}\PY{p}{,} \PY{l+m+mi}{0}\PY{p}{,} \PY{l+m+mi}{0}
	\PY{n}{krajni\PYZus{}zisk}\PY{p}{,} \PY{n}{krajni\PYZus{}i}\PY{p}{,} \PY{n}{krajni\PYZus{}delka} \PY{o}{=} \PY{l+m+mi}{0}\PY{p}{,} \PY{l+m+mi}{0}\PY{p}{,} \PY{l+m+mi}{0}
	\PY{k}{for} \PY{n}{i} \PY{o+ow}{in} \PY{n+nb}{range}\PY{p}{(}\PY{n}{n}\PY{p}{)}\PY{p}{:}
		\PY{k}{if} \PY{n}{krajni\PYZus{}zisk} \PY{o}{>} \PY{l+m+mi}{0}\PY{p}{:}        \PY{c}{\PYZsh{} Případ 2}
			\PY{n}{krajni\PYZus{}zisk} \PY{o}{+}\PY{o}{=} \PY{n}{h}\PY{p}{[}\PY{n}{i}\PY{p}{]}
			\PY{n}{krajni\PYZus{}delka} \PY{o}{+}\PY{o}{=} \PY{l+m+mi}{1}
		\PY{k}{else}\PY{p}{:}                      \PY{c}{\PYZsh{} Případ 3}
			\PY{n}{krajni\PYZus{}zisk} \PY{p}{,}\PY{n}{krajni\PYZus{}i}\PY{p}{,} \PY{n}{krajni\PYZus{}delka} \PY{o}{=} \PY{n}{h}\PY{p}{[}\PY{n}{i}\PY{p}{]}\PY{p}{,} \PY{n}{i}\PY{p}{,} \PY{l+m+mi}{1}

		\PY{k}{if} \PY{n}{krajni\PYZus{}zisk} \PY{o}{>} \PY{n}{max\PYZus{}zisk}\PY{p}{:} \PY{c}{\PYZsh{} Případ 1 (obráceně)}
			\PY{n}{max\PYZus{}zisk}\PY{p}{,} \PY{n}{max\PYZus{}i}\PY{p}{,} \PY{n}{max\PYZus{}delka} \PY{o}{=} \PY{n}{krajni\PYZus{}zisk}\PY{p}{,} \PY{n}{krajni\PYZus{}i}\PY{p}{,} \PY{n}{krajni\PYZus{}delka}
	\PY{k}{return} \PY{n}{max\PYZus{}zisk}\PY{p}{,} \PY{n}{max\PYZus{}i}\PY{p}{,} \PY{n}{max\PYZus{}delka}
\end{Verbatim}


\section{Implementace}
Program {\tt zisk.py} je v jazyce python 2.7.2. Vstupní data jsou kladná 
a záporná celá čísla. Výstupem je hodnota zisku, index první asistentky 
(od nuly), která začíná vybranou posloupnost a délka této posloupnosti. Jsou 
implementována řešení oběma metodami.

\subsection{Příklad použití}
\begin{verbatim}
# vygenerovani vstupnich dat:  300 cisel od -500 do 499
> test.data
for i in $(seq 1 300); do echo $(($RANDOM % 1000 - 500)) >> test.data; done

cat test.data | ./zisk.py
\end{verbatim}

\end{document}
